\chapter*{}
%\thispagestyle{empty}
%\cleardoublepage

%\thispagestyle{empty}

\input{portada/portada_2}



\cleardoublepage
\thispagestyle{empty}

\begin{center}
{\large\bfseries Detección de anomalías basada en técnicas de ensembles: Biblioteca de algoritmos}\\
\end{center}
\begin{center}
Ignacio Aguilera Martos\\
\end{center}

%\vspace{0.7cm}
\noindent{\textbf{Palabras clave}: anomalía, ensamblaje, python, hics, outres, loda, mahalanobis kernel, trinity, aprendizaje automático, estadística}\\

\vspace{0.7cm}
\noindent{\textbf{Resumen}}\\

La detección de anomalías es un ámbito de estudio que está ganando relevancia por ser una parte interesante en el tratamiento de los datos. Actualmente hacemos un manejo y un uso de los datos cada vez más voraz y creciente, necesitando no sólo técnicas que nos permitan obtener información de ellos sino además preprocesamiento de los datos que haga que estas técnicas funcionen de forma más eficiente. 

Las anomalías no sólo son útiles en el preprocesamiento de los datos, también son interesantes en detección de eventos atípicos en los mismos. Por ejemplo, podemos aplicar esta detección en casos como detección de fraude en compras con tarjetas bancarias o por ejemplo en predicción de fallos en sistemas como los frenos de un coche o un camión.

Para ello en el trabajo haremos un breve repaso de cuáles son las herramientas teóricas que hacen que el trabajo de nuestros algoritmos y modelos sea consistente y funcione así como herramientas estadísticas y del ámbito de la probabilidad que explican el funcionamiento de alguno de los modelos. 

Finalmente el trabajo desembocará en la implementación de algunos de los modelos del estado del arte en el ámbito de la detección de anomalías y su comparativa con modelos considerados como clásicos o tradicionales. Es decir, pondremos en contraposición los modelos de ensamblaje con los tradicionales para estudiarlos comparativamente. Asimismo veremos algunas conclusiones sobre los modelos y una propuesta de trabajo futuro con la intención de mejorar la situación actual de los modelos de ensamblaje.

\cleardoublepage


\thispagestyle{empty}


\begin{center}
{\large\bfseries Outlier detection based in ensemble methods: Library implementation}\\
\end{center}
\begin{center}
Ignacio Aguilera Martos\\
\end{center}

%\vspace{0.7cm}
\noindent{\textbf{Keywords}: outlier, ensemble, python, hics, outres, loda, mahalanobis kernel, trinity, machine learning, statistics}\\

\vspace{0.7cm}
\noindent{\textbf{Abstract}}\\

Write here the abstract in English.

\chapter*{Agradecimientos}
\thispagestyle{empty}

       \vspace{1cm}


Poner aquí agradecimientos...

