\chapter{Introducción}

Antes de comenzar el objeto de estudio de este trabajo, lo primero que debemos hacer es contextualizar el mismo y establecer un marco de trabajo en cuanto a teoría que se empleará en la posterior experimentación y desarrollo del mismo. 

El estudio realizado y plasmado en este trabajo se centra en la obtención de técnicas para la detección de anomalías en conjuntos de datos, concepto que introduciremos posteriormente. En concreto las técnicas que se van a desarrollar son las conocidas como técnicas de ensemble que se basan en el estudio del problema o bien combinando modelos existentes o bien haciendo un estudio pormenorizado aplicando algún criterio por subespacios del conjunto de datos. 

En primer lugar el trabajo desarrollará una introducción a la teoría de aprendizaje y resolución de problemas mediante datos y no por diseño así como la teoría matemática que esto involucra. Esta primera sección nos dará suficiente estructura al trabajo para poder definir en términos de distancias lo que significa que una instancia de un conjunto de datos sea una anomalía.

Posteriormente se desarrollará brevemente algunos conocimientos estadísticos básicos de estadística multivariante para poder introducir el concepto de anomalía desde la perspectiva de las probabilidades condicionadas.

Tras esto podremos entrar en el terreno de la experimentación, desarrollo y explicación de técnicas y puesta en contraste con los algoritmos clásicos para comprobar el desempeño de las nuevas técnicas.

Por último se presentarán las conclusiones obtenidas tras todo este estudio.