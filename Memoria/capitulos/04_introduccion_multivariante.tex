\chapter{Introducción de Estadística Multivariante}
\label{chapter:estadistica_multivariante}

Vamos a dar otra definición de anomalía que no coincide con la que hemos visto basada en distancias, pero antes de dar esa definición debemos hacer un breve repaso de estadística multivariante y probabilidad para poder comprender y enmarcar dicha definición.

\section{Introducción}

En primer lugar vamos a describir conceptos básicos sobre los que poder construir los conceptos que necesitamos para la definición de anomalía basada en probabilidades.

En primer lugar vamos a definir el concepto de variable aleatoria.

\begin{definicion}
	Una variable aleatoria es una función $X:\Omega \rightarrow E$ que parte de un espacio de probabilidad $(\Omega , \mathcal{F}, \mathcal{P})$ y llega a un espacio medible $(E, \mathcal{B})$, donde $X$ además es una función medible.
\end{definicion}

Normalmente ya sabemos que $E\subseteq \mathbb{R}$ y además cabe recordar que $\mathcal{F}$ es una $\sigma$-álgebra. Además cabe recordar la definición de función medible:

\begin{definicion}
	Decimos que una función $X: (\Omega , \mathcal{F}, \mathcal{P}) \rightarrow (E, \mathcal{B})$ es medible si $X^{-1}(B)\subset \mathcal{F}$, $\forall B \in \mathcal{B}$.
\end{definicion}

Esta definición puede extenderse al caso vectorial, introduciendo con esto la noción de vector aleatorio:

\begin{definicion}
	Un vector aleatorio $\underline{X} = (X_1 , ... , X_p)$ es una aplicación medible $\underline{X}: (\Omega , \mathcal{F}, \mathcal{P})\rightarrow (E, \mathcal{B}^p)$ donde $E\subseteq \mathbb{R}^p$.
\end{definicion}

Se puede demostrar además la caracterización:

\begin{proposicion}
	Un vector $\underline{X} = (X_1, ..., X_p)$ es un vector aleatorio si y sólo si $X_i : (\Omega , \mathcal{F}, \mathcal{P}) \rightarrow (\mathbb{R}, \mathcal{B})$ es una función medible.
\end{proposicion}

Con este vector aleatorio podemos estudiar o definir la distribución de probabilidad del mismo sobre $( \mathbb{R}^p , \mathcal{B}^p )$ $P_{\underline{X}}$ como:

$$P_{\underline{X}} [B]:= P[\underline{X}^{-1}(B)] \ \forall B\in \mathcal{B}$$

con lo que el espacio $(\mathbb{R}^p , \mathcal{B}^p , P_{\underline{X}})$ es un espacio de probabilidad o probabilístico.

Sobre los conocimientos de la definición de la función de distribución univariante podemos hacer una definición análoga para el caso multivariante.

\begin{definicion}
	Se define la función de distribución asociada a la probabilidad inducida como:
	
	$$F_{\underline{X}} (\underline{x}) = P_{\underline{X}} [X_1 \leq x_1 , ... , X_p \leq x_p] \ , \ \forall \underline{x} = (x_1 , ... , x_p) \in \mathbb{R}^p$$
\end{definicion}

De igual forma podemos caracterizar la función de densidad como aquella $f_{\underline{X}}$ que, de existir, cumple que:

$$F_{\underline{X}} (\underline{x}) = \int_{- \infty}^{x_1} \int_{-\infty}^{x_2} ... \int_{-\infty}^{x_p} f_{\underline{X}}(u_1 , ... , u_p) du_1 ... du_p$$

Otra forma de determinar de forma única la distribución de un vector aleatorio es mediante la función característica, lo que nos va a dar además una caracterización de la independencia que introduciremos en siguiente lugar.

\begin{definicion}
	Dado un vector aleatorio $X = (X_1 , ... , X_p)$ se define la función característica como $\Phi_{\underline{X}} (\underline{t}) = E[e^{i\underline{t}X}]$ con $\underline{t} = (t_1 , ... , t_p)\in \mathbb{R}^p$ donde la función $E[\cdot]$ denota la esperanza, por lo que:
	$$\Phi_{\underline{X}} (\underline{t}) = \int_{\mathbb{R}^p} e^{i\underline{t} \underline{X}} P_{\underline{X}}(d\underline{x})$$
\end{definicion}

Con esto ya podemos introducir el concepto de independencia en varias variables. 

\subsection{Independencia}

\begin{definicion}
	Dados dos vectores aleatorios $\underline{X} = (X_1 , ... , X_p)$, $\underline{Y} = (Y_1 , ... , Y_p)$ se dice que son independientes si:
	$$F_{\underline{X}, \underline{Y}}(\underline{x}, \underline{y}) = F_{\underline{X}}(\underline{x}) \cdot F_{\underline{Y}}(\underline{y})$$
\end{definicion}

Podemos también definir la independencia entre las variables de un vector aleatorio como:

\begin{definicion}
	$X = (X_1 , ... , X_p)$ se dice que está compuesto de variables independientes si $\forall B = B_1 \times ... \times B_p$ con $B_i \in \mathcal{B}$ se tiene que:
	
	$$P_{\underline{X}}(B) = P_{X_1}[B_1] \cdot ... \cdot P_{X_p}[B_p]$$
\end{definicion}

En cuanto a la independencia de sucesos podemos dar dos definiciones de independencia:

\begin{definicion}
	Decimos que los eventos $B = (B_1 , ... , B_p)$ son independientes dos a dos si para todos $m\neq k$ se tiene que $P(B_m \bigcap B_k) = P(B_m)P(B_k)$
\end{definicion}

\begin{definicion}
	Se dice que los eventos $B = (B_1 , ... , B_p)$ son independientes mutuamente si para todo $k\leq p$ se tiene que $P(\bigcap_{i=1}^{k}B_i) = \prod_{i=1}^{k}(B_i)$
\end{definicion}

En cuanto a la definición de independencia entre las variables aleatorias que definen un vector aleatorio podemos dar dos caracterizaciones basadas en la función característica.

\begin{proposicion}
	Si las componentes del vector aleatorio $X = (X_1 , ... , X_p)$ son independientes entonces:
	
	$$\Phi_{\underline{X}}(\underline{t}) = E[e^{i\underline{t}\underline{X}}] = \prod_{j=1}^{p}E[e^{it_j X_j}]$$
\end{proposicion}

\begin{proposicion}
	Si las componentes del vector aleatorio $X = (X_1 , ... , X_p)$ son independientes entonces la función característica de la variable $Y = \sum_{j=1}^{p}X_j$ es:
	
	$$\Phi_Y (t) = E[e^{itY}] = E[e^{it \sum_{j=1}^{p}X_j}] = \prod_{j=1}^{p}\Phi_{X_j} (t)$$
\end{proposicion}

\subsection{Probabilidad y esperanza condicionada}

En esta sección vamos a describir la probabilidad y esperanza condicionada de una variable aleatoria y no de un vector aleatorio. Este hecho es sencillo de deducir, pues como hemos introducido previamente la distribución de probabilidad de un vector aleatorio viene determinada por una distribución de probabilidad de una variable aleatoria. Por tanto el estudio de la probabilidad y esperanza condicionada en el caso univariante se hace válido para el caso multivariante. 

En primer lugar debemos introducir el concepto de probabilidad condicionada tal y cómo la conocemos hasta ahora de Bayes. Partimos de un espacio de probabilidad $(\Omega , \mathcal{A}, \mathcal{P})$.

\begin{definicion}
	Definimos la probabilidad condicionada a un suceso $B\in \mathcal{A}$ con $P(B)>0$ como:
	
	$$P(\cdot | B) : \mathcal{A} \rightarrow [0,1] , \ \ \ P(A | B) = \frac{P(A\cap B)}{P(B)}$$
\end{definicion}

Esta es una función de probabilidad, por lo que nos lleva a pensar en el espacio de probabilidad que genera, es más podemos pensar en el espacio de probabilidad en el que la probabilidad condicionada no se anula, es decir:

$$\mathcal{A}_B = \{ C = A\cap B, \ A\in \mathcal{A} \}$$

Por tanto solemos considerar como espacio de probabilidad condicionada al espacio $(B, \mathcal{A}_B , P(\cdot | B))$.

Partiendo de este espacio de probabilidad podemos considerar una variable aleatoria $X : (\Omega , \mathcal{A}, \mathcal{P}(\cdot | B)) \rightarrow (\mathbb{R}, \mathcal{B})$. 

\begin{definicion}
	Definimos la esperanza de esta variable aleatoria condicionada a $B$ como:
	$$E[X | B] = \int_{\Omega}XdP(\cdot | B) = \int_{\Omega}XdP(\cdot | B) = \frac{1}{P(B)}\int_{B}XdP = \frac{E[X1_B]}{P(B)}$$
	Donde $1_B$ representa la función indicadora del conjunto $B$.
\end{definicion}

No sólo podemos estudiar la probabilidad y esperanzas condicionadas a un evento, si no que también las podemos estudiar condicionadas a una $\sigma$-álgebra. En este terreno vamos a distinguir dos posibilidades: condicionamiento a una $\sigma$-álgebra generada por una partición numerable de sucesos de probabilidad no nula y condicionamiento a una $\sigma$-álgebra arbitraria.

\begin{definicion}
	Definimos la esperanza condicionada a una $\sigma$-álgebra $\mathcal{A}$ generada por $\{ B_n \}\subset \mathcal{A}$ con $B_i \cap B_j = \phi , \ i\neq j$, $\bigcup_{n=1}^{\infty} B_n = \Omega$ y $P(B_i)>0 , \ \forall i$. Siendo la $\mathcal{U} = \sigma (\{ B_n \})$ la $\sigma$-álgebra generada por $\{ B_n \}$. Con este marco, definimos la esperanza de una variable aleatoria $X: (\Omega , \mathcal{A}, P) \rightarrow (\mathbb{R}, \mathcal{B})$ condicionada a la $\sigma$-álgebra $\mathcal{U}$ como:
	
	$$E[X | \mathcal{U}](\omega) = \sum_{n=1}^{\infty} E[X | B_n]1_{B_n}(\omega)$$
\end{definicion}

\begin{propiedades}
	\begin{enumerate}
		\item $E[X | \mathcal{U}]: (\Omega , \mathcal{U}) \rightarrow (\mathbb{R}, \mathcal{B})$ es $\mathcal{U}$-medible.
		\item $E[E[X | \mathcal{U}]] = \sum_{n=1}^{\infty}E[X | B_n]P(B_n) = \sum_{n=1}^{\infty}E[X1_{B_n}] = E[X]$
	\end{enumerate}
\end{propiedades}

De igual forma podemos definir la probabilidad condicionada a una $\sigma$-álgebra generada por una partición numerable de sucesos no nulos.

\begin{definicion}
	Definimos la probabilidad de un suceso $A\in \mathcal{A}$ condicionada a la $\sigma$-álgebra $\mathcal{U}$ como:
	
	$$P(A | \mathcal{U}) = E[1_A | \mathcal{U}] = \sum_{n=1}^{\infty}E[1_A | B_n]1_{B_n} = \sum_{n=1}^{\infty}P(A | B_n)1_{B_n}$$ casi seguramente.
\end{definicion}

Podemos también dar unas propiedades inmediatas de la probabilidad condicionada tomando como base las de la esperanza.

\begin{propiedades}
	\begin{enumerate}
		\item $P(A | \mathcal{U})$ es $\mathcal{U}$-medible.
		\item $E[P(A | \mathcal{U})] = P(A)$
	\end{enumerate}
\end{propiedades}

Una vez visto esto podemos hacer una definición con una $\sigma$-álgebra arbitraria. Cabe decir que en este caso no vamos a poder dar una definición constructiva y fácil de calcular como sí hemos hecho en el caso particular anterior. Lo que sí vamos a tener con esta definición más general es el mantenimiento de las propiedades que hemos visto en primera instancia tanto de la probabilidad como de la esperanza condicionada. Sobra decir además que esta definición coincide con la anterior en el caso particular de una $\sigma$-álgebra generada por una partición numerable de sucesos no nulos.

\begin{definicion}
	Definimos la esperanza de una variable aleatoria $X$ en el marco dado condicionada a una $\sigma$-álgebra $\mathcal{U}\subset \mathcal{A}$ como la única función $\mathcal{U}$-medible tal que:
	
	$$\forall u \in \mathcal{U} \ \int_{\mathcal{U}}E[X | \mathcal{U}]P_{\mathcal{U}} = \int_{\mathcal{U}}X dP$$ casi seguramente $P_{\mathcal{U}}$. Donde $\forall u\in \mathcal{U}$ $P_{\mathcal{U}}(u) = P(u)$.
\end{definicion}

Igualmente podemos dar una definición de la probabilidad condicionada a una $\sigma$-álgebra arbitraria tomando como base la definición de esperanza condicionada.

\begin{definicion}
	Definimos la probabilidad de $A\in \mathcal{A}$ condicionada a la $\sigma$-álgebra $\mathcal{U}$ como:
	
	$$P(A | \mathcal{U}) = E[1_A | \mathcal{U}]$$ casi seguramente $P_{\mathcal{U}}$.
\end{definicion}

Por último antes de dar unas propiedades que nos den un poco más de conocimiento y herramientas de trabajo vamos a ver el concepto de probabilidad y esperanza condicionada a una variable aleatoria y no a un suceso o una $\sigma$-álgebra como hemos visto previamente.

Partimos igualmente del marco $(\Omega , \mathcal{A} , P)$ con dos variables aleatorias $X,Y$.

\begin{definicion}
	Definimos la $\sigma$-álgebra generada por la variable aleatoria $Y$ como la menor $\sigma$-álgebra que hace medible a la variable aleatoria $Y$ y la notaremos como $\sigma (Y)$.
\end{definicion}

Ahora si podemos definir la esperanza de una variable aleatoria condicionada a otra.

\begin{definicion}
	Definimos la esperanza de la variable aleatoria $X$ condicionada a la variable aleatoria $Y$ como:
	$$E[X | Y] = E[X | \sigma (Y)]$$
\end{definicion}

Como anotación cabe decir que esta esperanza condicionada es una función dependiente de la variable aleatoria $Y$, es decir podemos expresarla como:

$$g(y) = E[X | Y=y]$$

Ahora que tenemos la definición de la esperanza condicionada a una variable aleatoria podemos usar el concepto como hemos hecho anteriormente para definir la probabilidad de un suceso condicionado a una variable aleatoria.

\begin{definicion}
	Para todo $A\in \mathcal{A}$ definimos la probabilidad de $A$ condicionada a la variable aleatoria $Y$ como:
	$$P(A | Y) = E[1_A | \sigma (Y)]$$
	casi seguramente $P_{\sigma (Y)}$
\end{definicion}

Ahora estamos en condiciones de dar una propiedades elementales y de suavizamiento que nos van a dar herramientas con las esperanzas condicionadas. En este punto ya hemos visto que, al haber hecho las definiciones de esperanza y probabilidades usándolas indistintamente las propiedades que vamos a dar para la esperanza se pueden emplear para las probabilidades utilizando sus definiciones que impliquen el uso de esperanzas.

Sobre estas propiedades vamos a realizar algunas de las demostraciones de las propiedades elementales y de las de suavizamiento que vamos a dar para poner de relieve cómo podemos hacer uso de la probabilidad y esperanza condicionada.

\begin{propiedades}[Propiedades elementales]
	Partimos de un espacio de probabilidad $(\Omega , \mathcal{A}, P)$, $\mathcal{U}$ una $\sigma$-álgebra contenida en $\mathcal{A}$ y $X,Y$ variables aleatorias integrables.
	\begin{enumerate}
		\item $E[cte | \mathcal{U}] = cte$ casi seguramente $P_{\mathcal{U}}$
		\item Sean $a, b \in \mathbb{R}$ $E[aX + bY | \mathcal{U}] = aE[X | \mathcal{U}] + bE[Y | \mathcal{U}]$ casi seguramente $P_{\mathcal{U}}$, es decir, la esperanza condicionada cumple la propiedad de linealidad.
		\item $X\geq Y$ casi seguramente $P$ $\Rightarrow E[X | \mathcal{U}] \geq E[Y | \mathcal{U}]$ casi seguramente $P_{\mathcal{U}}$.
		\item $|E[X | \mathcal{U}]| \leq  E[|X| |\mathcal{U}]$
	\end{enumerate}
\end{propiedades}

\begin{demostracion}
	Vamos a demostrar la propiedad 1 para ver como trabajar con las igualdades casi seguras.
	\begin{enumerate}
		\item[1.] Como la igualdad es casi seguramente podemos aplicar integrales en la misma con lo que obtenemos lo siguiente:
		$$\forall u \in \mathcal{U} \ \int_{u} E[cte | \mathcal{U}]dP_{\mathcal{U}} = \int_{u}cte dP = cte P(u) = cte P_{\mathcal{U}}(u) = \int_{u}cte dP_{\mathcal{U}}$$
		Como la igualdad es con integrales, podemos decir por tanto que $E[cte | \mathcal{U}] = cte$ casi seguramente $P_{\mathcal{U}}$.
	\end{enumerate}
\end{demostracion}

\begin{propiedades}[Propiedades de suavizamiento]
	Partimos del marco del espacio probabilístico $(\Omega , \mathcal{A}, P)$ con una $\sigma$-álgebra $\mathcal{U}\subset \mathcal{A}$.
	\begin{enumerate}
		\item Si $X$ es una variable aleatoria integrable y $\mathcal{U}$-medible entonces se tiene que $E[X | \mathcal{U}] = X$ casi seguramente $P_{\mathcal{U}}$
		\item Sean $X, Y$ variables aleatorias con $X$ $\mathcal{U}$-medible, $Y$ integrable y $XY$ integrable, entonces se tiene que $E[XY | \mathcal{U}] = XE[Y | \mathcal{U}]$ casi seguramente $P_{\mathcal{U}}$.
		\item Se dice que $X$ es independiente de $\mathcal{U}$ si $X$ y $1_{\mathcal{U}}$ son independientes. Si $X$ es independiente de $\mathcal{U}$ entonces $E[X | \mathcal{U}] = E[X]$ casi seguramente $P_{\mathcal{U}}$.
		\item Sean $\mathcal{U}_1 \subset \mathcal{U}_2 \subset \mathcal{A}$ y $X$ una variable aleatoria integrable, entonces:
		$$E[X | \mathcal{U}_1] = E[E[X | \mathcal{U}_1] | \mathcal{U}_2] = E[E[X | \mathcal{U}_2] | \mathcal{U}_1]$$ casi seguramente $P_{\mathcal{U}}$.
	\end{enumerate}
\end{propiedades}

Vamos a hacer la demostración de las 4 propiedades para dar así una pincelada de cómo aplicar los conceptos vistos hasta ahora.

\begin{demostracion}
	Demostremos las propiedades de suavizamiento:
	\begin{enumerate}
		\item[4.] Sabemos que $E[X | \mathcal{U}_1] = Z$ es $\mathcal{U}_1$-medible y por tanto es $\mathcal{U}_2$-medible, por lo que $E[Z | \mathcal{U}_2] = Z$ casi seguramente $P_{\mathcal{U}_2}$.
		
		Vamos a utilizar ahora el hecho de que las igualdades son casi seguramente y por tanto vamos a ver si aplicando integrales en ambos lados de la igualdad obtenemos el mismo resultado y confirmamos la igualdad.
		
		$\forall u\in \mathcal{U}_1 \subset \mathcal{U}_2$ tenemos $\int_{u} E[E[X | \mathcal{U}_1] | \mathcal{U}_2]dP_{\mathcal{U}_2} = \int_{u}E[X | \mathcal{U}_1]dP_{\mathcal{U}_1} = \int_{u}XdP$
		
		Veamos ahora desarrollando el otro término.
		
		$\int_{u} E[E[X | \mathcal{U}_2] | \mathcal{U}_1]dP_{\mathcal{U}_1} = \int_{u}E[X | \mathcal{U}_2] dP_{\mathcal{U}_2} = \int_{u}XdP$
		
		Al haber llegado a la misma igualdad en integrales tenemos por tanto la igualdad casi seguramente que buscábamos.
		\item[1.] Como $X$ es $\mathcal{U}$-medible entonces tenemos que $\forall u \in \mathcal{U} \int_{u}E[X | \mathcal{U}] dP_{\mathcal{U}} = \int_{u}XdP = \int_{u}XdP_{\mathcal{U}}$ pues al ser $\mathcal{U}$-medible tenemos que $E[X] = \int_{\Omega} XdP = \int_{\Omega}XdP_{\mathcal{U}}$.
		\item[3.] $\forall u \in \mathcal{U}$ $\int_{u}E[X | \mathcal{U}]dP_{\mathcal{U}} = \int_{u} XdP = \int_{\Omega}1_{u}XdP = E[1_u X] = $
		
		$= E[1_u]E[X] = P(u)E[X] = P_{\mathcal{U}}(u)E[X] = \int_{u}E[X]dP_{\mathcal{U}}$
	\end{enumerate}
\end{demostracion}

Ya hemos dado las definiciones y propiedades de probabilidad y esperanza condicionadas, para finalizar vamos a ver algunas desigualdades famosas que utilizaremos y sus demostraciones.

\subsection{Desigualdades y fórmulas famosas}

\begin{teorema}[Desigualdad de Markov]
	Sea $X$ una variable aleatoria que toma valores no negativos. Entonces para cualquier constante $\alpha$ satisfaciendo $E[X]<\alpha$ se cumple que:
	
	$$P(X>\alpha) \leq \frac{E[X]}{\alpha}$$
\end{teorema}

\begin{demostracion}
	Denotemos como $f_{X}(x)$ la función de densidad de la variable aleatoria $X$. Entonces tenemos:
	
	$$E[X] = \int_{x}x f_{X}(x)dx = \int_{0\leq x\leq \alpha}xf_{X}(x)dx + \int_{x>\alpha}xf_{X}(x) dx$$
	
	$$\geq \int_{x>\alpha}xf_{X}(x)dx \geq \int_{x>\alpha}\alpha f_{X}(x)dx$$
	
	La primera de las desigualdades se sigue de la no negatividad de $X$ y la segunda se sigue de que la integral está definida sobre los puntos en los que $x>\alpha$, de hecho:
	
	$$\int_{x>\alpha}\alpha f_{X}(x)dx = \alpha P(X>\alpha)$$
	
	Con lo que tenemos finalmente que:
	
	$$E[X]\geq \alpha P(X>\alpha) \Leftrightarrow P(X>\alpha) \leq \frac{E[X]}{\alpha}$$ \QEDA
\end{demostracion}

\begin{teorema}[Desigualdad de Chebychev]
	Sea $X$ una variable aleatoria arbitraria. Entonces para cualquier constante $\alpha$ se tiene que:
	
	$$P(|X - E[X]|>\alpha)\leq \frac{Var[X]}{\alpha^2d}$$
\end{teorema}

\begin{demostracion}
	Sabemos que la desigualdad $|X-E[X]|>\alpha$ es cierta si y sólo si $|X-E[X]|^2 > \alpha^2$
	
	Vamos a definir la variable aleatoria $Y = (X-E[X])^2$ es es no negativa. Con esta definición se tiene que $E[Y] = Var[X]$ por la propia definición de la variable aleatoria $Y$.
	
	Entonces la parte izquierda de la desigualdad del teorema se puede expresar como $P(|X - E[X]|>\alpha) = P(Y>\alpha^2)$. Aplicando aquí la desigualdad de Markov obtenemos que:
	
	$$P(Y>\alpha^2) \leq \frac{E[Y]}{\alpha^2} = \frac{Var[X]}{\alpha^2}$$ \QEDA
\end{demostracion}

\begin{teorema}[Cota inferior de Chernoff]
	Sea $X$ una variable aleatoria que se puede expresar como la suma de $N$ variables aleatorias independiente de Bernoulli, cada una tomando el valor $1$ con probabilidad $p_i$.
	
	$$X = \sum_{i=1}^{N}X_i$$
	
	Entonces para todo $\delta \in (0,1)$ tenemos que:
	
	$$P(X<(1-\delta)E[X])<e^{-E[X]\delta^2 /2}$$
\end{teorema}

\begin{teorema}[Cota superior de Chernoff]
	Sea $X$ una variable aleatoria que se puede expresar como la suma de $N$ variables aleatorias independiente de Bernoulli, cada una tomando el valor $1$ con probabilidad $p_i$.
	
	$$X = \sum_{i=1}^{N}X_i$$
	
	Entonces para todo $\delta \in (0,2\cdot e -1)$ tenemos que:
	
	$$P(X>(1+\delta)E[X])<e^{-E[X]\delta^2 /2}$$
\end{teorema}

Ambas cotas disponen de una demostración que no es constructiva, por lo que no es relevante su demostración para el estudio. Como ejemplo de una demostración de un estilo similar haremos la demostración de la siguiente desigualdad.


\begin{teorema}[Desigualdad de Hoeffding]
	Sea $X$ una variable aleatoria que se puede expresar como suma de $N$ variables aleatorias independientes acotadas en intervalos $[l_i , u_i]$.
	
	$$X = \sum_{i=1}^{N}X_i$$
	
	Entonces para todo $\theta >0$ se tienen las cotas:
	
	$$P(X - E[X] > \theta) \leq e^{- \frac{2\theta^2}{\sum_{i=1}^{N}(u_i - l_i)^2}}$$
	
	$$P(E[X] - X > \theta) \leq e^{- \frac{2\theta^2}{\sum_{i=1}^{N}(u_i - l_i)^2}}$$
\end{teorema}

\begin{demostracion}
	Sólo haremos la demostración de la primera desigualdad de forma resumida y sin entrar en los detalles más complejos que se alejan del interés del estudio.
	
	En primer lugar debemos probar que para todo $t\geq 0$ se cumple la desigualdad:
	
	$$P(X - E[X]>\theta) = P(e^{t(X - E[X])}>e^{t\theta})$$
	
	Usando la desigualdad de Markov podemos probar que $P(e^{t(X-E[X])}>e^{t\theta})$ es como mucho $E[e^{(X-E[X])}]e^{-t\theta}$.
	
	Además al ser variables aleatorias independientes las que componen la variable aleatoria $X$ podemos descomponer el término teniendo la desigualdad:
	
	$$P(X - E[X]>\theta)\leq e^{-t\theta}\prod_{i}E[e^{t(X_i - E[X_i])}]$$
	
	Cada uno de los términos de este producto se puede probar que vale como mucho $e^{t^2(u_i - l_i)/8}$ usando argumentos de convexidad y el Teorema de Taylor.
	
	Por tanto se cumple:
	
	$$P(X-E[X]>\theta)\leq e^{-t\theta}\prod_{i}e^{t^2(u_i - l_i)^2 / 8}$$
	
	Nos interesa hallar el valor de $t=t^*$ que ajusta la desigualdad. Puede demostrarse que ese valor es:
	
	$$t^* = \frac{4\theta}{\sum_{i=1}^{N}(u_i - l_i)^2}$$
	
	Sustituyendo en la desigualdad con este valor de $t$ tenemos el resultado que queríamos probar. \QEDA
\end{demostracion}