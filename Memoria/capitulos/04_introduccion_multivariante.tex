\chapter{Introducción de Estadística Multivariante}
\label{chapter:estadistica_multivariante}

Vamos a dar otra definición de anomalía que no coincide con la que hemos visto basada en distancias, pero antes de dar esa definición debemos hacer un breve repaso de estadística multivariante y probabilidad para poder comprender y enmarcar dicha definición.

\section{Introducción}

En primer lugar vamos a describir conceptos básicos sobre los que poder construir los conceptos que necesitamos para la definición de anomalía basada en probabilidades.

En primer lugar vamos a definir el concepto de variable aleatoria.

\begin{definicion}
	Una variable aleatoria es una función $X:\Omega \rightarrow E$ que parte de un espacio de probabilidad $(\Omega , \mathcal{F}, \mathcal{P})$ y llega a un espacio medible $(E, \mathcal{B})$, donde $X$ además es una función medible.
\end{definicion}

Normalmente ya sabemos que $E\subseteq \mathbb{R}$ y además cabe recordar que $\mathcal{F}$ es una $\sigma$-álgebra. Además cabe recordar la definición de función medible:

\begin{definicion}
	Decimos que una función $X: (\Omega , \mathcal{F}, \mathcal{P}) \rightarrow (E, \mathcal{B})$ es medible si $X^{-1}(B)\subset \mathcal{F}$, $\forall B \in \mathcal{B}$.
\end{definicion}

Esta definición puede extenderse al caso vectorial, introduciendo con esto la noción de vector aleatorio:

\begin{definicion}
	Un vector aleatorio $\underline{X} = (X_1 , ... , X_p)$ es una aplicación medible $\underline{X}: (\Omega , \mathcal{F}, \mathcal{P})\rightarrow (E, \mathcal{B}^p)$ donde $E\subseteq \mathbb{R}^p$.
\end{definicion}

Se puede demostrar además la caracterización:

\begin{proposicion}
	Un vector $\underline{X} = (X_1, ..., X_p)$ es un vector aleatorio si y sólo si $X_i : (\Omega , \mathcal{F}, \mathcal{P}) \rightarrow (\mathbb{R}, \mathcal{B})$ es una función medible.
\end{proposicion}

Con este vector aleatorio podemos estudiar o definir la distribución de probabilidad del mismo sobre $( \mathbb{R}^p , \mathcal{B}^p )$ $P_{\underline{X}}$ como:

$$P_{\underline{X}} [B]:= P[\underline{X}^{-1}(B)] \ \forall B\in \mathcal{B}$$

con lo que el espacio $(\mathbb{R}^p , \mathcal{B}^p , P_{\underline{X}})$ es un espacio de probabilidad o probabilístico.

Sobre los conocimientos de la definición de la función de distribución univariante podemos hacer una definición análoga para el caso multivariante.

\begin{definicion}
	Se define la función de distribución asociada a la probabilidad inducida como:
	
	$$F_{\underline{X}} (\underline{x}) = P_{\underline{X}} [X_1 \leq x_1 , ... , X_p \leq x_p] \ , \ \forall \underline{x} = (x_1 , ... , x_p) \in \mathbb{R}^p$$
\end{definicion}

De igual forma podemos caracterizar la función de densidad como aquella $f_{\underline{X}}$ que, de existir, cumple que:

$$F_{\underline{X}} (\underline{x}) = \int_{- \infty}^{x_1} \int_{-\infty}^{x_2} ... \int_{-\infty}^{x_p} f_{\underline{X}}(u_1 , ... , u_p) du_1 ... du_p$$

Otra forma de determinar de forma única la distribución de un vector aleatorio es mediante la función característica, lo que nos va a dar además una caracterización de la independencia que introduciremos en siguiente lugar.

\begin{definicion}
	Dado un vector aleatorio $X = (X_1 , ... , X_p)$ se define la función característica como $\Phi_{\underline{X}} (\underline{t}) = E[e^{i\underline{t}X}]$ con $\underline{t} = (t_1 , ... , t_p)\in \mathbb{R}^p$ donde la función $E[\cdot]$ denota la esperanza, por lo que:
	$$\Phi_{\underline{X}} (\underline{t}) = \int_{\mathbb{R}^p} e^{i\underline{t} \underline{X}} P_{\underline{X}}(d\underline{x})$$
\end{definicion}

Con esto ya podemos introducir el concepto de independencia en varias variables. 