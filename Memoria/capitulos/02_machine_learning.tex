\part{Machine Learning y el concepto de Anomalía}
\label{part:machineLearning_anomalia}

En esta sección vamos a centrarnos en dos aspectos: el Machine Learning para establecer las herramientas usadas en el estudio y el propio concepto de anomalía y algunas reflexiones acerca del mismo.

\chapter{Machine Learning}
\label{chapter:machine_learning}

En este capítulo vamos a hacer un repaso sobre los conceptos asociados al Machine Learning, el aprendizaje y la teoría matemática que involucra. Estas herramientas y conceptos los utilizaremos posteriormente para resolver el problema de detección de anomalías.

\section{Contextualización del aprendizaje}

Para comenzar tenemos que empezar definiendo en que consiste el proceso de aprender sobre unos datos. Supongamos que tenemos un problema en el que tenemos una entrada y una salida, por ejemplo una entrada válida podría ser un vector $x\in \mathbb{R}^d$ y una salida un valor real o un número natural. El problema de aprendizaje intenta estimar una estructura de tipo entrada-salida como la descrita usando únicamente un número finito de observaciones.

Podemos definirlo de forma más general empleando tres conceptos:

\begin{itemize}
	\item Generador: El generador se encarga de obtener las entradas $x\in \mathbb{R}^d$ mediante una distribución de probabilidad $p(x)$ desconocida y fijada de antemano.
	\item Sistema: El sistema es el que produce la salida ``y'' (correcta) para cada entrada $x\in \mathbb{R}^d$ mediante la distribución de probabilidad $p(x|y)$ desconocida y fijada de antemano.
	\item Máquina de aprendizaje: esta es la que va a obtener información de las entradas y salidas conocidas para intentar predecir la salida correcta para una entrada nueva que se nos de. De forma abstracta esta máquina lo que hace es tomar una serie de funciones de un conjunto general de forma que para una entrada dada $x$ la función $f(x,\omega)$ con $\omega \in \Omega$ nos de la salida que corresponde para $x$ donde $\omega$ es una forma de indexar las funciones tomadas para generalizar la salida del conjunto más general de funciones que hemos indicado.
\end{itemize}

El único cabo que hemos dejado sin atar en las definiciones que acabamos de ver es el conjunto de funciones del cual tomaremos algunas para adaptar la máquina de aprendizaje a los datos recibidos. Este conjunto de funciones, que notaremos como $\mathcal{H}$, es de momento la única forma que tenemos de aplicar un conocimiento a priori en la máquina de aprendizaje.

Para finalizar esta breve introducción y poder continuar profundizando vamos a exponer algunos ejemplos de clases de funciones para que podamos visualizar el contexto.

\begin{itemize}
	\item Funciones lineales: En este caso la clase de funciones $\mathcal{H}$ está formada por funciones de la forma $h(x) = w_0 + \sum_{i=1}^{d}x_i w_i$ donde $w\in \mathbb{R}^{d+1}$. Este es el modelo de funciones más clásico.
	\item Funciones trigonométricas: Un ejemplo de una clase de funciones trigonométricas podría ser $f_m(x,v_m,w_m) = \sum_{j=1}^{m-1}(v_j \sin (jx) + w_j \cos (jx)) + w_0$ donde en este caso la entrada es un único valor real. Este tipo de clases de funciones serán útiles en problemas de regresión que luego explicaremos con algo más de detalle aunque sin centrarnos mucho en ello pues no es el objetivo del estudio.
\end{itemize}

\subsection{Objetivo del aprendizaje}

Cuando hablamos de aprendizaje queremos obtener algo de dicho aprendizaje a partir de los datos. Como ya se ha mencionado, intentamos obtener una función de una familia de funciones que aproxime o modele de buena manera la salida del sistema. Por tanto, ese es nuestro objetivo: obtener una función de la familia de funciones que minimice el error.

El problema que enfrentamos es que sólo disponemos de un número finito, por ejemplo $n$, de observaciones de datos y su correspondiente salida. Esto nos va a hacer que no podamos tener una garantía de optimalidad a no ser que tendamos $n$ a infinito. 

Sin embargo si que podemos cuantificar cómo de buena es una aproximación con respecto a otra mediante la función pérdida o error que denotaremos como $L(y,f(x,\omega))$. Esta función nos va a medir la diferencia entre la salida real del sistema y la salida dada por la función $f$ para la entrada $x$ siendo siempre $L(y,f(x,\omega))\geq 0$.

Recordemos además que el Generador obtiene datos mediante una distribución desconocida pero fijada de antemano y que son independientes e idénticamente distribuidos con respecto a la distribución conjunta, es decir:

$$p(x,y) = p(x)p(y|x)$$

Una vez definido todo esto podemos obtener el valor esperado de pérdida o error mediante el funcional

$$R(\omega) = \int L(y,f(x,\omega))p(x,y)dxdy$$

Ahora podemos concretar un poco más lo que entendemos como objetivo del aprendizaje. El objetivo será encontrar una función $f\in \mathcal{H}$ que nos minimice el valor del funcional $R(\omega)$. Pero recordemos que $p(x,y)$ es desconocida para nosotros, por lo que no podemos saber cómo se distribuyen los datos y por tanto el valor del funcional no es calculable para nosotros y por tanto la solución puramente de cálculo no es accesible.

Por tanto, la única forma realmente potente y útil de encontrar una buena aproximación será incorporar el conocimiento a priori que tenemos del sistema. En la sección anterior hemos visto que una forma de incorporar dicho conocimiento es mediante la selección de la clase de funciones, pero además será muy relevante el hecho de cómo los datos son empleados en el proceso de aprendizaje. En este apartado de decisión tendremos que resolver primero la codificación de los datos, el algoritmo empleado y el uso de técnicas como la regularización que veremos después para incorporar nuestro conocimiento en el camino que nos lleve a la solución.

\subsection{Clases de aprendizaje}

El problema de aprendizaje puede ser subdividido a su vez en cuatro clases distintas y que se suelen abordar de forma independiente. Estoss tipos de problemas de aprendizaje son:

\begin{itemize}
	\item Clasificación: El problema de clasificación consiste en identificar y separar instancias de datos según su clase. Por ejemplo podemos dividir a la población mundial en dos clases: sanos y enfermos. Un problema de clasificación podría ser saber identificar estas clases para un conjunto de personas. Los problemas de clasificación más sencillos son aquellos en los que se usan dos únicas clases aunque se puede generalizar la definición del problema a k-clases.
	\item Regresión: El problema de regresión consiste en estimar una función $f: \mathbb{R}^n \rightarrow \mathbb{R}$ a partir de una serie de muestras previas con los valores de $f$. Un problema de regresión podría ser determinar la función que, dados los datos de altura y dimensiones corporales sea capaz de darnos el peso aproximado de la persona.
	\item Estimación de la función de densidad: en este caso no nos interesa la salida que proporciona el sistema, ya sea el valor de una clase o una función real como en el caso de la regresión. En este caso el objetivo del aprendizaje es conseguir la función de densidad $f(x,\omega)$, con $\omega \in \Omega$ los parámetros necesarios de la función de densidad, con la que se distribuyen los datos de entrada del sistema.
	\item Agrupamiento y cuantificación vectorial: El problema de cuantificación vectorial consiste en intentar explicar la distribución de los vectores de entrada mediante puntos clave llamados centroides. De esta forma se podría reducir la complejidad de los datos expresándolos en función de un sistema de generadores menor. El problema de agrupamiento tiene también relación por utilizar la idea de centroide, pero el objetivo es completamente distinto. El objetivo del problema de agrupamiento es intentar conseguir agrupar los datos en clusters, es decir, regiones del espacio en las que se concentran un conjunto de datos. De esta forma intentamos agrupar los datos que mantienen una relación entre sí. Un ejemplo de un problema de cuantificación vectorial podría ser un problema de reducción de dimensionalidad y un ejemplo de problema de agrupamiento podría ser identificar instancias de datos con características comunes.
\end{itemize}

\section{Principios y adaptación del aprendizaje}

Según Vapnik \cite{vapnik_v._nature_nodate} la predicción mediante el aprendizaje se puede dividir en dos fases:

\begin{enumerate}
	\item Aprendizaje o estimación a partir de una muestra.
	\item Predicción a partir de las estimaciones obtenidas.
\end{enumerate}

Estas dos fases se corresponden con los dos tipos de inferencia clásica que conocemos, esto es, inducción y deducción. Traído a este caso el proceso de inducción es aquel que a partir de los datos de aprendizaje o los datos de la muestra que tenemos con la salida que corresponde podemos estimar un modelo. Es decir, estamos sacando el conocimiento de los datos para generar el modelo. El proceso de deducción es aquel que, una vez obtenido el modelo estimado (la generalización) obtenemos una predicción de la salida sobre un conjunto de datos.

Por contra, Vapnik propone un paso que resuelve estas dos fases directamente y que él denomina transducción. Este paso consiste en, dados los datos de entrenamiento obtenemos directamente los valores de salida sin tener que hacer la generalización a un modelo. De esta forma, según Vapnik, podríamos reducir el error que cometemos en la predicción. Este razonamiento tiene sentido, pues estamos omitiendo el paso más complejo del proceso de inducción-deducción.

En resumen esta idea se puede resumir en la siguiente figura:

\begin{figure}[H]
	\centering
	\includegraphics[scale=0.5]{imagenes/induccion_deduccion_transduccion}
	\label{ind_ded_trans}
	\caption{Tipos de inferencia y transduccion \cite[p.~41]{cherkassky_learning_2007}}
\end{figure}

Podemos ver que el conocimiento a priori que tenemos del problema se manifiesta una vez se crea el modelo general, de forma que se emplearía en el paso de la inducción. Ya hemos hablado previamente del conocimiento a priori y cómo incorporarlo al modelo, pero por concretar un poco más podemos añadirlo básicamente de dos formas:

\begin{itemize}
	\item Escogiendo un conjunto de funciones para aproximar la salida del sistema
	\item Añadiendo restricciones o penalizaciones adicionales a dicho conjunto de funciones.
\end{itemize}

En resumen, para poder crear la generalización del modelo de forma única necesitamos:

\begin{enumerate}
	\item Un conjunto de funciones para aproximar la salida.
	\item Conocimiento a priori.
	\item Un principio inductivo, que no es más que una indicación de cómo emplear los datos para llegar a la generalización del modelo.
	\item Un método de aprendizaje, es decir, una implementación del principio inductivo.
\end{enumerate}

En secciones posteriores revisaremos algunos de los principios inductivos más usados pero es importante reseñar la diferencia entre pricipio inductivo y método de aprendizaje. Para un mismo principio inductivo podemos tener varios métodos de aprendizaje, pues podemos escoger diferentes formas de llevarlo a la práctica. Por ejemplo, uno de los principios inductivos más empleados es el ERM o Empirical Risk Minimization, es decir, minimización del error empírico. Podríamos pensar en diferentes formas de utilizar este principio, por ejemplo sólo avanzamos en la creación del modelo si a cada paso que demos minimizamos el error, o por ejemplo vamos avanzando varios modelos a la vez hasta obtener un número de modelos finales de entre los cuales escogeremos aquel que mejor minimice dicho error.

\subsection{Principios inductivos}

Una vez introducido el concepto como hemos hecho en la sección anterior vamos a hacer un breve repaso de los principios más usados y en qué consiste cada uno de ellos.

\subsubsection{Penalización o Regularización}

Imaginemos que tenemos una clase de funciones muy flexible, esto es con un gran número de parámetros libres $f(x,\omega)$ con $\omega \in \Omega$. Vamos a partir de la base del ERM, es decir, minimizar el error empírico. La penalización lo que va a hacer es añadir un factor a la función a minimizar:

$$R_{pen}(\omega) = R_{emp}(\omega) + \lambda \phi [f(x,\omega)]$$

Donde $R_{emp}(\omega)$ es el error empírico con los parámetros $\omega$ y $\phi [f(x,\omega)]$ es un funcional no negativo asociado a cada estimación $f(x,\omega)$. El parámetro $\lambda >0$ es un escalar que controla el peso de la penalización.

El funcional $\phi [f(x,\omega)]$ puede medir lo que creamos conveniente que debemos añadir, es decir, aquí podemos añadir a la minimización algún tipo de medida que nos diga cómo de bien funciona el ajuste de los datos y cómo de bien funciona la información a priori que hemos incluido en el modelo. Pensemos por ejemplo que $\lambda$ fuera un parámetro con un valor muy alto. En este caso la penalización por un mal ajuste de los datos no sería de gran importancia pues lo más conveniente sería minimizar el valor del funcional para no obtener una gran penalización. De esta forma podemos ajustar y dar un poco más de información al error empírico. Por ejemplo, en función del problema, es posible medir la complejidad de la solución mediante el funcional $\phi$ y de esta forma no sólo vamos a obtener una función que ajuste bien los datos, si no que también mantenga una cierta simplicidad para evitar por ejemplo el sobreajuste.

\subsubsection{Reglas de parada anticipada}

Pensemos en un método que vaya aprendiendo de los datos de forma iterativa intentando a cada iteración reducir el error cometido, por ejemplo el ERM. Los métodos o reglas de parada anticipada pueden verse como penalizaciones sobre el algoritmo conforme se va ejecutando. Las reglas de parada anticipada, como su nombre indica lo que preveen es la parada del algoritmo antes de obtener su objetivo teórico. Por ejemplo un algoritmo intenta que el error sea menor que $10^{-6}$ pero para reducirlo desde $10^{-4}$ hasta $10^{-5}$ está consumiendo millones de iteraciones. Si queremos que el tiempo de cómputo penalice lo que podemos hacer es fijar por ejemplo un número máximo de iteraciones que detenga el método aunque no se haya alcanzado esa barrera de error que se preveía.

\subsubsection{Minimización del riesgo estructural o SRM}

Para entender esta filosofía nos ponemos en la situación de que ya sabemos la clase de funciones con la que vamos a aproximar la salida del sistema, por ejemplo hemos escogido la clase de funciones polinómicas. Bajo esta clase de funciones podemos ordenar las funciones por complejidad, entendiendo por complejidad el número de parámetros de la función. Por ejemplo los polinomios de grado $m$ son de menor complejidad que los de grado $m+1$. De esta forma podemos pensar en una estructura de la clase de funciones de la forma:

$$S_0 \subset S_1 \subset S_2 \subset \cdots$$

Este parámetro de complejidad también puede ser un principio a minimizar para intentar conseguir una solución adecuada pero también simple. La generalización de la medida de complejidad para las clases de funciones es la conocida como dimensión VC o dimensión de Vapnik-Chervonenkis.

\subsubsection{Inferencia Bayesiana}

Este principio inductivo se utiliza en el problema de estimación de la función de densidad. El principio es utilizar la conocida fórmula de Bayes para hacer una estimación de la función de densidad empleando el conocimiento a priori que disponemos del problema. La forma en la que se emplea esta fórmula es de la siguiente:

$$P[modelo | datos] = \frac{P[datos | modelo] \cdot P[modelo]}{P[datos]}$$

donde $P[modelo]$ es la probabilidad a priori, $P[datos]$ es la probabilidad de los datos de entrenamiento y $P[datos | modelo]$ es la probabilidad de que los datos estén generados por el modelo.

\subsubsection{Descripción de mínima longitud}

La idea de este principio es la minimización de la longitud que se necesita emplear para describir un modelo y la correspondiente salida. Llamamos l a la longitud total:

$$l = L(modelo) + L(datos | modelo)$$

Esta medida puede ser vista como una medida de complejidad conjunta de todo el modelo.

\section{Regularización}

Por la importancia de este principio inductivo vamos a desarrollarlo un poco más, junto con el concepto de penalización, la selección de los modelos y la relación entre sesgo y varianza. Este último es un concepto muy relevante en cuanto al aprendizaje y que en nuestro caso, al no poseer la clasificación real tendremos que tenerlo en cuenta.

\subsection{Problema de la alta dimensionalidad}

Sabemos que cuando estamos ante un problema de aprendizaje nuestro objetivo es conseguir estimar una función con un número finito de instancias de una muestra ya con la salida. Al tener un número finito de elementos en la muestra ya sabemos que no podemos garantizar que la respuesta sea la óptima o correcta, pero además debemos pensar que a mayor regularidad del conjunto de funciones empleado debemos tener una densidad suficiente de puntos para compensar dicha regularidad. Este problema es conocido como la maldición de la dimensionalidad (curse of dimensionality). El problema es que cuanto mayor sea la dimensionalidad considerada más difícil es poder tener esa alta densidad de datos que se requieren para funciones muy regulares.

Este problema que conlleva la alta dimensionalidad proviene de la geometría de los espacio con alta dimensionalidad. A medida que incrementamos la dimensionalidad el espacio se ve cada vez con más aristas o picos. Podemos pensar en un cubo para el espacio tridimensional y a medida que aumentamos la dimensión incorporamos más aristas y vértices. Podemos resumir en 4 propiedades de los espacio con alta dimensionalidad que causan este problema:

\begin{enumerate}
	\item La densidad disminuye exponencialmente al aumentar el número de dimensiones. Supongamos que tenemos una muestra de $n$ puntos en $\mathbb{R}$. Para poder tener la misma densidad en un espacio $d$-dimensional $\mathbb{R}^d$ necesidamos $n^d$ puntos.
	\item Cuanto mayor dimensionalidad tenga el conjunto de datos mayor lado se necesita para que un hipercubo contenga el mismo porcentaje del conjunto que con una menor dimensionalidad. Imaginemos que tenemos un conjunto $d$-dimensional en el que tenemos la muestra dentro de un hipercubo unidad. Si quisiéramos abarcar un porcentaje $p\in [0,1]$ necesitaríamos un cubo de lado $e_d (p) = p^{\frac{1}{d}}$. Como se puede observar a mayor dimensionalidad y $p$ constante el lado es cada vez mayor. Esta idea es fácilemente entendible si observamos la siguiente figura:
	
	\begin{figure}[H]
		\centering
		\label{radio_alta_dimensionalidad}
		\includegraphics[scale=0.6]{imagenes/radio_alta_dimensionalidad}
		\caption{Para 2 dimensiones necesitamos menor lado que para 3 dimensiones. \cite[p.~64]{cherkassky_learning_2007}}
	\end{figure}
	\item Casi todo punto está más cerca de un borde que de otro punto. Pensemos en un conjunto de datos con $n$ puntos distribuidos de forma uniforme en una bola $d$-dimensional de radio unidad. Para este conjunto de datos, según \cite{hastie_t._elements_nodate}, la distancia media entre el centro de la distribución y los puntos más cercanos a dicho centro se mide bajo la fórmula:
	
	$$D(d,n) = (1-\frac{1}{2}^{1/n})^{1/d}$$
	
	Si en esta fórmula tomamos por ejemplo $n=200$ y $d=10$ el resultado es $D(10,200) \approx 0.57$. Esto significa que los puntos más cercanos al centro de la distribución están más cerca de los bordes que del centro.
	\item Casi todo punto es una anomalía sobre su propia proyección. Si pensamos de nuevo en la idea de los vértices y aristas en espacio de alta dimensionalidad y pensamos en que, según el punto anterior, cada vez que aumenta la dimensionalidad los puntos están más cerca de los bordes entonces no es extraño pensar que los puntos a medida que aumenta la dimensionalidad están más distantes del resto de puntos. Esto intuitivamente (ya que aún no hemos visto la definición formal de anomalía) nos guía a pensar que vistos los puntos en sus propios entornos éstos serán anomalías comparados con el resto.
	
	\begin{figure}[H]
		\centering
		\label{espacio_alta_dimension}
		\includegraphics[scale=0.6]{imagenes/espacio_alta_dimension}
		\caption{Forma conceptual de un espacio de alta dimensionalidad.\cite[p.~64]{cherkassky_learning_2007}}
	\end{figure}

	Conceptualmente podemos imaginarlo con esta forma de picos, con lo que si tenemos los datos apiñados en dichos picos o extremos el resto de datos que estén en picos diferentes distan tanto del que estamos considerando que no podemos afirmar que tengan ninguna relación entre sí.
\end{enumerate}

Estos puntos hemos de recordar que van referidos al conjunto de datos y no a las funciones que estamos considerando para representar la salida del sistema. Si estamos considerando la complejidad de las funciones la dimensionalidad no es una buena medida. Sabemos de la existencia de teoremas de aproximación de funciones como por ejemplo el Teorema de Superposición de Kolmogorov-Arnold.

\begin{teorema}[Teorema de Superposición de Kolmogorov-Arnold]
	Sea $f$ una función continua de varias variables $f:X_1 \times ... \times X_n \rightarrow \mathbb{R}$, entonces existen funciones $\Phi_q : \mathbb{R}\rightarrow \mathbb{R}$ y $\phi_{q,p} : X_p \rightarrow [0,1]$ tales que $f$ se puede expresar como:
	
	$$f(x) = f(x_1, ..., x_n) = \sum_{q=0}^{2n}\Phi_q ( \sum_{p=1}^{n}\phi_{q,p}(x_p))$$
\end{teorema}

Este Teorema argumenta perfectamente que la complejidad que le damos a los datos por tener una alta dimensionalidad no es transferible a las funciones pues podemos expresar funciones de varias variables como combinación de funciones de una sola variables. En otras palabras no podemos argumentar que la complejidad de funciones univariantes sea mayor o menor que la de funciones multivariantes.

\chapter{Concepto de anomalía}
\label{chapter:anomalia}