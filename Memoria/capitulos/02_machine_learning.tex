\part{Machine Learning y el concepto de Anomalía}
\label{part:machineLearning_anomalia}

En esta sección vamos a centrarnos en dos aspectos: el Machine Learning para establecer las herramientas usadas en el estudio y el propio concepto de anomalía y algunas reflexiones acerca del mismo.

\chapter{Machine Learning}
\label{chapter:machine_learning}

En este capítulo vamos a hacer un repaso sobre los conceptos asociados al Machine Learning, el aprendizaje y la teoría matemática que involucra. Estas herramientas y conceptos los utilizaremos posteriormente para resolver el problema de detección de anomalías.

\section{Contextualización del aprendizaje}

Para comenzar tenemos que empezar definiendo en que consiste el proceso de aprender sobre unos datos. Supongamos que tenemos un problema en el que tenemos una entrada y una salida, por ejemplo una entrada válida podría ser un vector $x\in \mathbb{R}^d$ y una salida un valor real o un número natural. El problema de aprendizaje intenta estimar una estructura de tipo entrada-salida como la descrita usando únicamente un número finito de observaciones.

Podemos definirlo de forma más general empleando tres conceptos:

\begin{itemize}
	\item Generador: El generador se encarga de obtener las entradas $x\in \mathbb{R}^d$ mediante una distribución de probabilidad $p(x)$ desconocida y fijada de antemano.
	\item Sistema: El sistema es el que produce la salida ``y'' (correcta) para cada entrada $x\in \mathbb{R}^d$ mediante la distribución de probabilidad $p(x|y)$ desconocida y fijada de antemano.
	\item Máquina de aprendizaje: esta es la que va a obtener información de las entradas y salidas conocidas para intentar predecir la salida correcta para una entrada nueva que se nos de. De forma abstracta esta máquina lo que hace es tomar una serie de funciones de un conjunto general de forma que para una entrada dada $x$ la función $f(x,\omega)$ con $\omega \in \Omega$ nos de la salida que corresponde para $x$ donde $\omega$ es una forma de indexar las funciones tomadas para generalizar la salida del conjunto más general de funciones que hemos indicado.
\end{itemize}

El único cabo que hemos dejado sin atar en las definiciones que acabamos de ver es el conjunto de funciones del cual tomaremos algunas para adaptar la máquina de aprendizaje a los datos recibidos. Este conjunto de funciones, que notaremos como $\mathcal{H}$, es de momento la única forma que tenemos de aplicar un conocimiento a priori en la máquina de aprendizaje.

Para finalizar esta breve introducción y poder continuar profundizando vamos a exponer algunos ejemplos de clases de funciones para que podamos visualizar el contexto.

\begin{itemize}
	\item Funciones lineales: En este caso la clase de funciones $\mathcal{H}$ está formada por funciones de la forma $h(x) = w_0 + \sum_{i=1}^{d}x_i w_i$ donde $w\in \mathbb{R}^{d+1}$. Este es el modelo de funciones más clásico.
	\item Funciones trigonométricas: Un ejemplo de una clase de funciones trigonométricas podría ser $f_m(x,v_m,w_m) = \sum_{j=1}^{m-1}(v_j \sin (jx) + w_j \cos (jx)) + w_0$ donde en este caso la entrada es un único valor real. Este tipo de clases de funciones serán útiles en problemas de regresión que luego explicaremos con algo más de detalle aunque sin centrarnos mucho en ello pues no es el objetivo del estudio.
\end{itemize}

\chapter{Concepto de anomalía}
\label{chapter:anomalia}