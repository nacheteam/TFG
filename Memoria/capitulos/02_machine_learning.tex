\part{Machine Learning y el concepto de Anomalía}
\label{part:machineLearning_anomalia}

En esta sección vamos a centrarnos en dos aspectos: el Machine Learning para establecer las herramientas usadas en el estudio y el propio concepto de anomalía y algunas reflexiones acerca del mismo.

\chapter{Machine Learning}
\label{chapter:machine_learning}

En este capítulo vamos a hacer un repaso sobre los conceptos asociados al Machine Learning, el aprendizaje y la teoría matemática que involucra. Estas herramientas y conceptos los utilizaremos posteriormente para resolver el problema de detección de anomalías.

\section{Contextualización del aprendizaje}

Para comenzar tenemos que empezar definiendo en que consiste el proceso de aprender sobre unos datos. Supongamos que tenemos un problema en el que tenemos una entrada y una salida, por ejemplo una entrada válida podría ser un vector $x\in \mathbb{R}^d$ y una salida un valor real o un número natural. El problema de aprendizaje intenta estimar una estructura de tipo entrada-salida como la descrita usando únicamente un número finito de observaciones.

Podemos definirlo de forma más general empleando tres conceptos:

\begin{itemize}
	\item Generador: El generador se encarga de obtener las entradas $x\in \mathbb{R}^d$ mediante una distribución de probabilidad $p(x)$ desconocida y fijada de antemano.
	\item Sistema: El sistema es el que produce la salida ``y'' (correcta) para cada entrada $x\in \mathbb{R}^d$ mediante la distribución de probabilidad $p(x|y)$ desconocida y fijada de antemano.
	\item Máquina de aprendizaje: esta es la que va a obtener información de las entradas y salidas conocidas para intentar predecir la salida correcta para una entrada nueva que se nos de. De forma abstracta esta máquina lo que hace es tomar una serie de funciones de un conjunto general de forma que para una entrada dada $x$ la función $f(x,\omega)$ con $\omega \in \Omega$ nos de la salida que corresponde para $x$ donde $\omega$ es una forma de indexar las funciones tomadas para generalizar la salida del conjunto más general de funciones que hemos indicado.
\end{itemize}

El único cabo que hemos dejado sin atar en las definiciones que acabamos de ver es el conjunto de funciones del cual tomaremos algunas para adaptar la máquina de aprendizaje a los datos recibidos. Este conjunto de funciones, que notaremos como $\mathcal{H}$, es de momento la única forma que tenemos de aplicar un conocimiento a priori en la máquina de aprendizaje.

Para finalizar esta breve introducción y poder continuar profundizando vamos a exponer algunos ejemplos de clases de funciones para que podamos visualizar el contexto.

\begin{itemize}
	\item Funciones lineales: En este caso la clase de funciones $\mathcal{H}$ está formada por funciones de la forma $h(x) = w_0 + \sum_{i=1}^{d}x_i w_i$ donde $w\in \mathbb{R}^{d+1}$. Este es el modelo de funciones más clásico.
	\item Funciones trigonométricas: Un ejemplo de una clase de funciones trigonométricas podría ser $f_m(x,v_m,w_m) = \sum_{j=1}^{m-1}(v_j \sin (jx) + w_j \cos (jx)) + w_0$ donde en este caso la entrada es un único valor real. Este tipo de clases de funciones serán útiles en problemas de regresión que luego explicaremos con algo más de detalle aunque sin centrarnos mucho en ello pues no es el objetivo del estudio.
\end{itemize}

\subsection{Objetivo del aprendizaje}

Cuando hablamos de aprendizaje queremos obtener algo de dicho aprendizaje a partir de los datos. Como ya se ha mencionado, intentamos obtener una función de una familia de funciones que aproxime o modele de buena manera la salida del sistema. Por tanto, ese es nuestro objetivo: obtener una función de la familia de funciones que minimice el error.

El problema que enfrentamos es que sólo disponemos de un número finito, por ejemplo $n$, de observaciones de datos y su correspondiente salida. Esto nos va a hacer que no podamos tener una garantía de optimalidad a no ser que tendamos $n$ a infinito. 

Sin embargo si que podemos cuantificar cómo de buena es una aproximación con respecto a otra mediante la función pérdida o error que denotaremos como $L(y,f(x,\omega))$. Esta función nos va a medir la diferencia entre la salida real del sistema y la salida dada por la función $f$ para la entrada $x$ siendo siempre $L(y,f(x,\omega))\geq 0$.

Recordemos además que el Generador obtiene datos mediante una distribución desconocida pero fijada de antemano y que son independientes e idénticamente distribuidos con respecto a la distribución conjunta, es decir:

$$p(x,y) = p(x)p(y|x)$$

Una vez definido todo esto podemos obtener el valor esperado de pérdida o error mediante el funcional

$$R(\omega) = \int L(y,f(x,\omega))p(x,y)dxdy$$

Ahora podemos concretar un poco más lo que entendemos como objetivo del aprendizaje. El objetivo será encontrar una función $f\in \mathcal{H}$ que nos minimice el valor del funcional $R(\omega)$. Pero recordemos que $p(x,y)$ es desconocida para nosotros, por lo que no podemos saber cómo se distribuyen los datos y por tanto el valor del funcional no es calculable para nosotros y por tanto la solución puramente de cálculo no es accesible.

Por tanto, la única forma realmente potente y útil de encontrar una buena aproximación será incorporar el conocimiento a priori que tenemos del sistema. En la sección anterior hemos visto que una forma de incorporar dicho conocimiento es mediante la selección de la clase de funciones, pero además será muy relevante el hecho de cómo los datos son empleados en el proceso de aprendizaje. En este apartado de decisión tendremos que resolver primero la codificación de los datos, el algoritmo empleado y el uso de técnicas como la regularización que veremos después para incorporar nuestro conocimiento en el camino que nos lleve a la solución.

\subsection{Clases de aprendizaje}

El problema de aprendizaje puede ser subdividido a su vez en cuatro clases distintas y que se suelen abordar de forma independiente. Estoss tipos de problemas de aprendizaje son:

\begin{itemize}
	\item Clasificación: El problema de clasificación consiste en identificar y separar instancias de datos según su clase. Por ejemplo podemos dividir a la población mundial en dos clases: sanos y enfermos. Un problema de clasificación podría ser saber identificar estas clases para un conjunto de personas. Los problemas de clasificación más sencillos son aquellos en los que se usan dos únicas clases aunque se puede generalizar la definición del problema a k-clases.
	\item Regresión: El problema de regresión consiste en estimar una función $f: \mathbb{R}^n \rightarrow \mathbb{R}$ a partir de una serie de muestras previas con los valores de $f$. Un problema de regresión podría ser determinar la función que, dados los datos de altura y dimensiones corporales sea capaz de darnos el peso aproximado de la persona.
	\item Estimación de la función de densidad: en este caso no nos interesa la salida que proporciona el sistema, ya sea el valor de una clase o una función real como en el caso de la regresión. En este caso el objetivo del aprendizaje es conseguir la función de densidad $f(x,\omega)$, con $\omega \in \Omega$ los parámetros necesarios de la función de densidad, con la que se distribuyen los datos de entrada del sistema.
	\item Agrupamiento y cuantificación vectorial: El problema de cuantificación vectorial consiste en intentar explicar la distribución de los vectores de entrada mediante puntos clave llamados centroides. De esta forma se podría reducir la complejidad de los datos expresándolos en función de un sistema de generadores menor. El problema de agrupamiento tiene también relación por utilizar la idea de centroide, pero el objetivo es completamente distinto. El objetivo del problema de agrupamiento es intentar conseguir agrupar los datos en clusters, es decir, regiones del espacio en las que se concentran un conjunto de datos. De esta forma intentamos agrupar los datos que mantienen una relación entre sí. Un ejemplo de un problema de cuantificación vectorial podría ser un problema de reducción de dimensionalidad y un ejemplo de problema de agrupamiento podría ser identificar instancias de datos con características comunes.
\end{itemize}

\chapter{Concepto de anomalía}
\label{chapter:anomalia}