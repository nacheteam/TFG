\chapter{Concepto de anomalía}
\label{chapter:anomalia}

\section{Contextualización}

Ya hemos discutido previamente una idea intuitiva del concepto de anomalía. Un dato decimos que es anómalo cuando se distancia del resto de los datos lo suficiente como para no tener características comunes con el resto.

Este hecho puede ser por distintos motivos. Puede que la anomalía venga del hecho de que se está produciendo un evento en nuestros experimentos que no sea nada frecuente. Por ejemplo podemos estar midiendo datos meteorológicos y que en un momento dado se den una serie de fenómenos que no sea frecuente ver juntos, o incluso que no se hayan visto nunca ocurrir simultáneamente. Otra forma de tener una anomalía en nuestro conjunto de datos pudiera ser errores de medición. Por ejemplo si seguimos con este símil de los datos meteorológicos imaginemos que nuestra estación dispone de un termómetro. Este sensor se ha roto y empieza a marcar datos superiores a 100 $C^\circ$, claramente son datos muy desviados de las temperaturas normales con lo que no tendrían relación con el resto y presentaría una desviación muy importante con respecto al resto de los datos.

