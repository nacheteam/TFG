\documentclass[a4paper,11pt]{book}
%\documentclass[a4paper,twoside,11pt,titlepage]{book}
\usepackage{listings}
\usepackage[utf8]{inputenc}
\usepackage[spanish]{babel}

% \usepackage[style=list, number=none]{glossary} %
%\usepackage{titlesec}
%\usepackage{pailatino}

\decimalpoint
\usepackage{dcolumn}
\newcolumntype{.}{D{.}{\esperiod}{-1}}
\makeatletter
\addto\shorthandsspanish{\let\esperiod\es@period@code}
\makeatother


%\usepackage[chapter]{algorithm}
\RequirePackage{verbatim}
%\RequirePackage[Glenn]{fncychap}
\usepackage{fancyhdr}
\usepackage{graphicx}
\usepackage{afterpage}

\usepackage{float}
\usepackage[spanish,onelanguage]{algorithm2e} %for psuedo code

% Math style letters
\usepackage{amsfonts}
\usepackage{amsmath}
\usepackage{amssymb}

\usepackage{longtable}

\usepackage[pdfborder={000}]{hyperref} %referencia


%%%%%%%%%%%%%%%%%%%%%%%%%%%%%%%%%%%%%%%%%%%%%%%%%%%%%%%%%%%%%%%%%%%%%%%%%%%%%%%%%%%%%%%%%%%%%%%%%%
%%                                 Para codigo python                                           %%
%%%%%%%%%%%%%%%%%%%%%%%%%%%%%%%%%%%%%%%%%%%%%%%%%%%%%%%%%%%%%%%%%%%%%%%%%%%%%%%%%%%%%%%%%%%%%%%%%%

\usepackage{color}
\usepackage{listings}
\usepackage{setspace}
\definecolor{Code}{rgb}{0,0,0}
\definecolor{Decorators}{rgb}{0.5,0.5,0.5}
\definecolor{Numbers}{rgb}{0.5,0,0}
\definecolor{MatchingBrackets}{rgb}{0.25,0.5,0.5}
\definecolor{Keywords}{rgb}{0,0,1}
\definecolor{self}{rgb}{1,0,0.5}
\definecolor{Strings}{rgb}{0,0.63,0}
\definecolor{Comments}{rgb}{0,0.63,1}
\definecolor{Backquotes}{rgb}{0,0,0}
\definecolor{Classname}{rgb}{1,0,0.5}
\definecolor{FunctionName}{rgb}{0,0,1}
\definecolor{Operators}{rgb}{1,0,0.5}
\definecolor{Background}{rgb}{0.98,0.98,0.98}
\lstdefinelanguage{Python}{
	numbers=left,
	numberstyle=\footnotesize,
	numbersep=1em,
	xleftmargin=1em,
	framextopmargin=2em,
	framexbottommargin=2em,
	showspaces=false,
	showtabs=false,
	showstringspaces=false,
	frame=l,
	tabsize=4,
	% Basic
	basicstyle=\ttfamily\small\setstretch{1},
	backgroundcolor=\color{Background},
	% Comments
	commentstyle=\color{Comments}\sffamily,
	% Strings
	stringstyle=\color{Strings},
	morecomment=[s][\color{Strings}]{"""}{"""},
	morecomment=[s][\color{Strings}]{'''}{'''},
	% keywords
	morekeywords={import,from,class,def,for,while,if,is,in,elif,else,not,and,or,print,break,continue,return,True,False,None,access,as,,del,except,exec,finally,global,import,lambda,pass,print,raise,try,assert},
	keywordstyle={\color{Keywords}\bfseries},
	% additional keywords
	morekeywords={[2]@invariant,pylab,numpy,np,scipy},
	keywordstyle={[2]\color{Decorators}\slshape},
	emph={self},
	emphstyle={\color{self}\slshape},
	%
}

%%%%%%%%%%%%%%%%%%%%%%%%%%%%%%%%%%%%%%%%%%%%%%%%%%%%%%%%%%%%%%%%%%%%%%%%%%%%%%%%%%%%%%%%%%%%%%%%%%

% ********************************************************************
% Re-usable information
% ********************************************************************
\newcommand{\myTitle}{Biblioteca de algoritmos de detección de anomalías basados en técnicas de ensembles\xspace}
\newcommand{\myDegree}{Doble grado Ingeniería Informática y Matemáticas\xspace}
\newcommand{\myName}{Ignacio Aguilera Martos\xspace}
\newcommand{\myProf}{Francisco Herrera Triguero\xspace}
\newcommand{\myOtherProf}{Jacinto Carrasco Castillo\xspace}
%\newcommand{\mySupervisor}{Put name here\xspace}
\newcommand{\myFaculty}{Escuela Técnica Superior de Ingenierías Informática y de
Telecomunicación y Facultad de Ciencias\xspace}
\newcommand{\myFacultyShort}{E.T.S. de Ingenierías Informática y de
Telecomunicación y Facultad de Ciencias\xspace}
\newcommand{\myDepartment}{Departamento de Ciencias de la Computación e Inteligencia Artificial\xspace}
\newcommand{\myUni}{\protect{Universidad de Granada}\xspace}
\newcommand{\myLocation}{Granada\xspace}
\newcommand{\myTime}{\today\xspace}
\newcommand{\myVersion}{Version 0.1\xspace}


\hypersetup{
pdfauthor = {\myName (nacheteam@correo.ugr.es)},
pdftitle = {\myTitle},
pdfsubject = {},
pdfkeywords = {outlier, anomalía, ensemble},
pdfcreator = {LaTeX},
pdfproducer = {pdflatex}
}

%\hyphenation{}


%\usepackage{doxygen/doxygen}
%\usepackage{pdfpages}
\usepackage{url}
\usepackage{colortbl,longtable}
\usepackage[stable]{footmisc}
%\usepackage{index}

%\makeindex
%\usepackage[style=long, cols=2,border=plain,toc=true,number=none]{glossary}
% \makeglossary

% Definición de comandos que me son tiles:
%\renewcommand{\indexname}{Índice alfabético}
%\renewcommand{\glossaryname}{Glosario}

\pagestyle{fancy}
\fancyhf{}
\fancyhead[LO]{\leftmark}
\fancyhead[RE]{\rightmark}
\fancyhead[RO,LE]{\textbf{\thepage}}
\renewcommand{\chaptermark}[1]{\markboth{\textbf{#1}}{}}
\renewcommand{\sectionmark}[1]{\markright{\textbf{\thesection. #1}}}

\setlength{\headheight}{1.5\headheight}

\newcommand{\HRule}{\rule{\linewidth}{0.5mm}}
%Definimos los tipos teorema, ejemplo y definición podremos usar estos tipos
%simplemente poniendo \begin{teorema} \end{teorema} ...
\newtheorem{teorema}{Teorema}[chapter]
\newtheorem{proposicion}{Proposición}[chapter]
\newtheorem{propiedad}{Propiedad}[chapter]
\newtheorem{lema}{Lema}[chapter]
\newtheorem{demostracion}{Demostración}[chapter]
\newtheorem{propiedades}{Propiedades}[chapter]
\newtheorem{ejemplo}{Ejemplo}[chapter]
\newtheorem{definicion}{Definición}[chapter]

\newcommand*{\QEDA}{\hfill\ensuremath{\blacksquare}}%
\newcommand*{\QEDB}{\hfill\ensuremath{\square}}%

\definecolor{gray97}{gray}{.97}
\definecolor{gray75}{gray}{.75}
\definecolor{gray45}{gray}{.45}
\definecolor{gray30}{gray}{.94}

\lstset{ frame=Ltb,
     framerule=0.5pt,
     aboveskip=0.5cm,
     framextopmargin=3pt,
     framexbottommargin=3pt,
     framexleftmargin=0.1cm,
     framesep=0pt,
     rulesep=.4pt,
     backgroundcolor=\color{gray97},
     rulesepcolor=\color{black},
     %
     stringstyle=\ttfamily,
     showstringspaces = false,
     basicstyle=\scriptsize\ttfamily,
     commentstyle=\color{gray45},
     keywordstyle=\bfseries,
     %
     numbers=left,
     numbersep=6pt,
     numberstyle=\tiny,
     numberfirstline = false,
     breaklines=true,
   }
 
% minimizar fragmentado de listados
\lstnewenvironment{listing}[1][]
   {\lstset{#1}\pagebreak[0]}{\pagebreak[0]}

\lstdefinestyle{CodigoC}
   {
	basicstyle=\scriptsize,
	frame=single,
	language=C,
	numbers=left
   }
\lstdefinestyle{CodigoC++}
   {
	basicstyle=\small,
	frame=single,
	backgroundcolor=\color{gray30},
	language=C++,
	numbers=left
   }

 
\lstdefinestyle{Consola}
   {basicstyle=\scriptsize\bf\ttfamily,
    backgroundcolor=\color{gray30},
    frame=single,
    numbers=none
   }


\newcommand{\bigrule}{\titlerule[0.5mm]}


%Para conseguir que en las páginas en blanco no ponga cabecerass
\makeatletter
\def\clearpage{%
  \ifvmode
    \ifnum \@dbltopnum =\m@ne
      \ifdim \pagetotal <\topskip
        \hbox{}
      \fi
    \fi
  \fi
  \newpage
  \thispagestyle{empty}
  \write\m@ne{}
  \vbox{}
  \penalty -\@Mi
}
\makeatother

\usepackage{pdfpages}
\begin{document}
\setlength{\parskip}{10pt}

\begin{titlepage}
 
 
\newlength{\centeroffset}
\setlength{\centeroffset}{-0.5\oddsidemargin}
\addtolength{\centeroffset}{0.5\evensidemargin}
\thispagestyle{empty}

\noindent\hspace*{\centeroffset}\begin{minipage}{\textwidth}

\centering
\includegraphics[width=0.9\textwidth]{imagenes/logos/logo_ugr.jpg}\\[1.4cm]

\textsc{ \Large TRABAJO FIN DE GRADO\\[0.2cm]}
\textsc{ Doble Grado Ingeniería Informática y Matemáticas}\\[1cm]
% Upper part of the page
% 
% Title
{\Huge\bfseries Detección de anomalías basada en técnicas de ensembles\\
}
\noindent\rule[-1ex]{\textwidth}{3pt}\\[3.5ex]
{\large\bfseries Biblioteca de algoritmos}
\end{minipage}

\vspace{2.5cm}
\noindent\hspace*{\centeroffset}\begin{minipage}{\textwidth}
\centering

\textbf{Autor}\\ {Ignacio Aguilera Martos}\\[2.5ex]
\textbf{Directores}\\
{Francisco Herrera Triguero\\
Jacinto Carrasco Castillo}\\[2cm]
\includegraphics[width=0.3\textwidth]{imagenes/logos/etsiit_logo.png}\\[0.1cm]
\textsc{Escuela Técnica Superior de Ingenierías Informática y de Telecomunicación}\\
\includegraphics[width=0.2\textwidth]{imagenes/logos/ciencias.png}\\[0.1cm]
\textsc{Facultad de Ciencias}\\
\textsc{---}\\
Granada, mes de 201
\end{minipage}
%\addtolength{\textwidth}{\centeroffset}
%\vspace{\stretch{2}}
\end{titlepage}



\chapter*{}
%\thispagestyle{empty}
%\cleardoublepage

%\thispagestyle{empty}

\input{portada/portada_2}



\cleardoublepage
\thispagestyle{empty}

\begin{center}
{\large\bfseries Detección de anomalías basada en técnicas de ensembles: Biblioteca de algoritmos}\\
\end{center}
\begin{center}
Ignacio Aguilera Martos\\
\end{center}

%\vspace{0.7cm}
\noindent{\textbf{Palabras clave}: anomalía, ensamblaje, python, hics, outres, loda, mahalanobis kernel, trinity, aprendizaje automático, estadística, probabilidad, multivariante}\\

\vspace{0.7cm}
\noindent{\textbf{Resumen}}\\

La detección de anomalías es un ámbito de estudio que está ganando relevancia por ser una parte interesante en el tratamiento de los datos. Actualmente hacemos un manejo y un uso de los datos cada vez más voraz y creciente, necesitando no sólo técnicas que nos permitan obtener información de ellos sino además preprocesamiento de los datos que haga que estas técnicas funcionen de forma más eficiente. 

Las anomalías no sólo son útiles en el preprocesamiento de los datos, también son interesantes en detección de eventos atípicos en los mismos. Por ejemplo, podemos aplicar esta detección en casos como detección de fraude en compras con tarjetas bancarias o por ejemplo en predicción de fallos en sistemas como los frenos de un coche o un camión.

Para ello en el trabajo haremos un breve repaso de cuáles son las herramientas teóricas que hacen que el trabajo de nuestros algoritmos y modelos sea consistente y funcione así como herramientas estadísticas y del ámbito de la probabilidad que explican el funcionamiento de alguno de los modelos. 

Finalmente el trabajo desembocará en la implementación de algunos de los modelos del estado del arte en el ámbito de la detección de anomalías y su comparativa con modelos considerados como clásicos o tradicionales. Es decir, pondremos en contraposición los modelos de ensamblaje con los tradicionales para estudiarlos comparativamente. Asimismo veremos algunas conclusiones sobre los modelos y una propuesta de trabajo futuro con la intención de mejorar la situación actual de los modelos de ensamblaje.

\cleardoublepage


\thispagestyle{empty}


\begin{center}
{\large\bfseries Outlier detection based in ensemble methods: Library implementation}\\
\end{center}
\begin{center}
Ignacio Aguilera Martos\\
\end{center}

%\vspace{0.7cm}
\noindent{\textbf{Keywords}: outlier, ensemble, python, hics, outres, loda, mahalanobis kernel, trinity, machine learning, statistics, probability, multivariate}\\

\vspace{0.7cm}
\noindent{\textbf{Abstract}}\\

Nowadays it's being more and more important the analysis of outliers in the area of data preprocessing. The way we are using the data is more and more greedy and we need better ways of mining the knowledge from the data. The use of outlier detection can lead to better performance of the Machine Learning models as we can now detect the outliers and eliminate them or treat them separately. Another approach is the detection of certain events due to the appearance of outliers. For example we can have a record of a certain bank and the credit card users. If someone gets his card stolen the way the thief buys and where the payments occur can lead to a detection of an anomalous way of using this card and therefore to inform the person that he could have his card stolen.

The anomaly detection could be useful as well in various field as medicine or event detection. We could think and measure the failure of the brake system of a truck or an earthquake as an anomaly in our datasets and we could detect them or even predict them. In medicine we could see the problem as finding what's wrong on a set of patients. We could think for example on a set of blood tests. One of them is very bad in terms of some of the values. If it is posed against healthy people the anomaly detection algorithm could give us not only the fact that the person is ill but also where the anomalous values are coming from to make it easier for the doctor to discover the problem.

The aim of this project is the discussion of ensemble models against the conventional ones in the discussion of the dilema of meta-algorithms versus simple algorithms. This study accomplishes the task of theoretical and practical discussion of the topic as well as further conclusions towards future work.

The way we introduced the outlier detection and study on this job has been preceded by a deep study on Machine Learning and the statistics it involves. This previous study is due to the importance of understanding why the algorithms and models implemented in this job are suitable for their use and therefore we can know as well their limits. This section has a description and introduction of the Machine Learning area and it evolves on the study of ERM and the in sample and out sample dilema in the field. A review on bias variance tradeoff is also given in this section. We need to consider that the problem we are handling is a non-supervised problem as in the general scenario we will not have the gound truth available, this is, the labels indicating which instance is an outlier and which isn't. That's why it is important to have a review and understand that variance and bias are put in a two hand balance, if we reduce variance we increase bias and viceversa.

After this study the first concept of outlier is given based on distances and quartiles. It is not only important the way we define the outliers but also why should we consider detecting them on our datasets before getting any further work done. Criterions and possibilities are described in this section as for example the Tukey's Fences criterion which is the most used one.

Probability, and even more multivariate probability, is very important in our study. Nearly all models implemented base their working principle on density or probability estimation. For this reason I've considered getting our hands firs on a multivariate introduction with useful concepts for explaining and understanding the model. This does not means that all concepts are going to be used in the study but, those included have been useful to me in the process of understanding how the theory of the models is built around. This section gets some introduction concepts of random vectors taking into account the definition and properties of random variables. It also touches the independence concept and some caracterizations involving random vectors. The most useful and interesting part of this section from the mathematical point of view is the conditional probability and conditional expectation. These concepts are relevant because they have not been introduced during the course of the degree and they are very important an related with Stochastic Processes. These concepts can apply to dynamically generated data as time series or stream flow data.

Due to this probability introduction a new outlier definition is available to us through these concepts. A probability-based definition of outlier is given and then used in serveral models as HICS or OUTRES. This probability definition involves probability density expectation, marginal distribution and joint distribution. This definition is not put against the distance based one, but to complement of fullfill the first definition with non trivial outliers. This concept is drawn from the OUTRES and HICS papers.

All this theory gives us way to approach the model implementation. The concept of ensemble appears here as the algorithms used in this study are no conventional ones. As the title says, the study goes around outlier ensembles. The concept is given here as the final goal is to make a reasonable comparison between the outlier detection algorithms, those called ensembles and what we could consider as traditional or conventional. We can consider a ensemble algorithm as an algorithm that makes use of simple algorithms combining them or algorithms that make a data transformation to study the dataset in a specific way. This includes not only doing projections to make for example histograms but also the study of the data in subspaces and applying different techniques or selection mechanisms depending on the subspace. The definition is made by giving the different types of ensembles based on Charu's Aggarwal book on outlier detection and outlier ensembles.

This portfolio contains the implementation and explanation of the five models chosen: HICS, OUTRES, Trinity, Mahalanobis Kernel and LODA. They are not chosen by coincidence but trying to cover most of the types of algorithms available in the state of art of outlier ensembles. HICS and OUTRES are from the subtype of ensembles called subspace miners of subspace based. The basic explanation of them is that they try to analyze the data in certain subspaces so they can measure the density in each one of them and compare it to the rest of the instances of the dataset. HICS mantaing the point of view of choosing the subspaces with no relation or conection with the instances of the dataset while OUTRES tries the other way around, this is choosing the subspace based on the instance we are considering. Mahalanobis Kernel is a PCA-based algorithm or at least PCA-influenced as it belongs to the same category. Trinity is a meta-model or meta-algorithm, this is a algorithm that combines simple models in order to obtain a more robust algorithm. Finally LODA uses the last technique I've chosen which is histogram-based. The short explanation of this type of model is the use of histograms to study the frequency of appearance of a certain value in the data.

So now we have all set up for our experiments. For the implementation Python has been chosen for several reasons. First of all Python is a versatile flexible language giving a lot of possibilities on the libraries you can use. Second we can make the development usefull for more people as Python is one of the most used languages on data science. For this reason I've decided to make myself a library with the algorithms implemented so they can be easily used by Python users and furthermore the development is completely open and available in GitHub so that anyone can see the code, understand it and fix bugs or extend the library content with new models if necessary.

All models are put against the same set of datasets given from the Stony Brooks University. This University mantains a set of datasets for outlier detection with the ground truth available so practitioners can work with it. These datasets allow us to treat the problem as a semi-supervised problem as we can obtain at the end a percentage of accuracy. The work first executes all models on these datasets to measure the performance of the ensemble models. To compare them with the classical ones we need the implementation of those classical. For this purpose the library PyOD is used. This library contains the implementation of a lot of models including some ensemble algorithms but not the ones I've chosen. With these implementations we can compare the accuracy and times of the models over the datasets. For a better understanding of the models they are analyzed individually if necessary.

So, I have ended up with a study of the state of art of ensemble algorithms putting them against the classical approach to see who outperforms whom. Future work is also discused at the end of this portfolio as the possibilities of improvement are still there due to the difficulty of the problem and the novelty of the ensemble thechniques.

\chapter*{Agradecimientos}
\thispagestyle{empty}

En primer lugar me gustaría agradecer a mi tutor, Francisco Herrera Triguero por su apoyo en el desarrollo de este trabajo. Ha sido un honor poder desarrollar este trabajo con su ayuda. Igualmente me gustaría agradecerle a Jacinto Carrasco Castillo su ayuda en el desarrollo del mismo y su dedicación conmigo.

Me gustaría agradecer el apoyo también de mis padres quienes me han acompañado durante estos 5 años y sin los cuales no podría haber completado este camino. Gracias por ser referentes para mí y no dejarme desistir en mis momentos más bajos.

A mis amigos y compañeros que han seguido el mismo camino que yo y que tanto apoyo me han dado para completar estas carreras. Luis, Pablo, Antonio, Diego, Darío, Iñaki y todos los que no puedo mencionar por no hacer estos agradecimientos infinitos. Gracias a todos por ser no sólo unos compañeros si no unos verdaderos amigos. También quiero agradecer a mis amigos fuera de la carrera. Alberto, Alex, Luismi, Pablo, Carmen, Elio y todos los que me dejo en el tintero, gracias por ayudarme y darme un empujón siempre que lo he necesitado.

A mi prima María y mi primo Fran quienes también me han servido de apoyo en esta carrera. A mis abuelos, Ramiro, Vicenta y María. No he podido tener unos mejores referentes que vosotros. Espero llegar a poder ser así algún día. Espero también haber cumplido las expectativas que pudierais tener sobre mí.

Y a tí, María. Gracias por hacer este camino lo más fácil posible, por tus ánimos, tu cariño y comprensión. Me has enseñado lo que es verdadero trabajo duro, perseverancia y mejorar como persona. No puedo más que agradecerte estos años juntos y espero me dejes seguir acompañándote y formando un equipo.

Gracias a todos, os quiero.

\frontmatter
\tableofcontents
%\listoffigures
%\listoftables
%
\mainmatter

\chapter{Introducción}

Antes de comenzar el objeto de estudio de este trabajo, lo primero que debemos hacer es contextualizar el mismo y establecer un marco de trabajo en cuanto a teoría que se empleará en la posterior experimentación y desarrollo del mismo. 

El estudio realizado y plasmado en este trabajo se centra en la obtención de técnicas para la detección de anomalías en conjuntos de datos, concepto que introduciremos posteriormente. En concreto las técnicas que se van a desarrollar son las conocidas como técnicas de ensemble que se basan en el estudio del problema o bien combinando modelos existentes o bien haciendo un estudio pormenorizado aplicando algún criterio por subespacios del conjunto de datos. 

En primer lugar el trabajo desarrollará una introducción a la teoría de aprendizaje y resolución de problemas mediante datos y no por diseño así como la teoría matemática que esto involucra. Esta primera sección nos dará suficiente estructura al trabajo para poder definir en términos de distancias lo que significa que una instancia de un conjunto de datos sea una anomalía.

Posteriormente se desarrollará brevemente algunos conocimientos estadísticos básicos de estadística multivariante para poder introducir el concepto de anomalía desde la perspectiva de las probabilidades condicionadas.

Tras esto podremos entrar en el terreno de la experimentación, desarrollo y explicación de técnicas y puesta en contraste con los algoritmos clásicos para comprobar el desempeño de las nuevas técnicas.

Por último se presentarán las conclusiones obtenidas tras todo este estudio.
%
\part{Machine Learning y el concepto de Anomalía}
\label{part:machineLearning_anomalia}

En esta sección vamos a centrarnos en dos aspectos: el Machine Learning para establecer las herramientas usadas en el estudio y el propio concepto de anomalía y algunas reflexiones acerca del mismo.

\chapter{Machine Learning}
\label{chapter:machine_learning}

En este capítulo vamos a hacer un repaso sobre los conceptos asociados al Machine Learning, el aprendizaje y la teoría matemática que involucra. Estas herramientas y conceptos los utilizaremos posteriormente para resolver el problema de detección de anomalías.

\section{Contextualización del aprendizaje}

Para comenzar tenemos que empezar definiendo en que consiste el proceso de aprender sobre unos datos. Supongamos que tenemos un problema en el que tenemos una entrada y una salida, por ejemplo una entrada válida podría ser un vector $x\in \mathbb{R}^d$ y una salida un valor real o un número natural. El problema de aprendizaje intenta estimar una estructura de tipo entrada-salida como la descrita usando únicamente un número finito de observaciones.

Podemos definirlo de forma más general empleando tres conceptos:

\begin{itemize}
	\item Generador: El generador se encarga de obtener las entradas $x\in \mathbb{R}^d$ mediante una distribución de probabilidad $p(x)$ desconocida y fijada de antemano.
	\item Sistema: El sistema es el que produce la salida ``y'' (correcta) para cada entrada $x\in \mathbb{R}^d$ mediante la distribución de probabilidad $p(x|y)$ desconocida y fijada de antemano.
	\item Máquina de aprendizaje: esta es la que va a obtener información de las entradas y salidas conocidas para intentar predecir la salida correcta para una entrada nueva que se nos de. De forma abstracta esta máquina lo que hace es tomar una serie de funciones de un conjunto general de forma que para una entrada dada $x$ la función $f(x,\omega)$ con $\omega \in \Omega$ nos de la salida que corresponde para $x$ donde $\omega$ es una forma de indexar las funciones tomadas para generalizar la salida del conjunto más general de funciones que hemos indicado.
\end{itemize}

El único cabo que hemos dejado sin atar en las definiciones que acabamos de ver es el conjunto de funciones del cual tomaremos algunas para adaptar la máquina de aprendizaje a los datos recibidos. Este conjunto de funciones, que notaremos como $\mathcal{H}$, es de momento la única forma que tenemos de aplicar un conocimiento a priori en la máquina de aprendizaje.

Para finalizar esta breve introducción y poder continuar profundizando vamos a exponer algunos ejemplos de clases de funciones para que podamos visualizar el contexto.

\begin{itemize}
	\item Funciones lineales: En este caso la clase de funciones $\mathcal{H}$ está formada por funciones de la forma $h(x) = w_0 + \sum_{i=1}^{d}x_i w_i$ donde $w\in \mathbb{R}^{d+1}$. Este es el modelo de funciones más clásico.
	\item Funciones trigonométricas: Un ejemplo de una clase de funciones trigonométricas podría ser $f_m(x,v_m,w_m) = \sum_{j=1}^{m-1}(v_j \sin (jx) + w_j \cos (jx)) + w_0$ donde en este caso la entrada es un único valor real. Este tipo de clases de funciones serán útiles en problemas de regresión que luego explicaremos con algo más de detalle aunque sin centrarnos mucho en ello pues no es el objetivo del estudio.
\end{itemize}

\subsection{Objetivo del aprendizaje}

Cuando hablamos de aprendizaje queremos obtener algo de dicho aprendizaje a partir de los datos. Como ya se ha mencionado, intentamos obtener una función de una familia de funciones que aproxime o modele de buena manera la salida del sistema. Por tanto, ese es nuestro objetivo: obtener una función de la familia de funciones que minimice el error.

El problema que enfrentamos es que sólo disponemos de un número finito, por ejemplo $n$, de observaciones de datos y su correspondiente salida. Esto nos va a hacer que no podamos tener una garantía de optimalidad a no ser que tendamos $n$ a infinito. 

Sin embargo si que podemos cuantificar cómo de buena es una aproximación con respecto a otra mediante la función pérdida o error que denotaremos como $L(y,f(x,\omega))$. Esta función nos va a medir la diferencia entre la salida real del sistema y la salida dada por la función $f$ para la entrada $x$ siendo siempre $L(y,f(x,\omega))\geq 0$.

Recordemos además que el Generador obtiene datos mediante una distribución desconocida pero fijada de antemano y que son independientes e idénticamente distribuidos con respecto a la distribución conjunta, es decir:

$$p(x,y) = p(x)p(y|x)$$

Una vez definido todo esto podemos obtener el valor esperado de pérdida o error mediante el funcional

$$R(\omega) = \int L(y,f(x,\omega))p(x,y)dxdy$$

Ahora podemos concretar un poco más lo que entendemos como objetivo del aprendizaje. El objetivo será encontrar una función $f\in \mathcal{H}$ que nos minimice el valor del funcional $R(\omega)$. Pero recordemos que $p(x,y)$ es desconocida para nosotros, por lo que no podemos saber cómo se distribuyen los datos y por tanto el valor del funcional no es calculable para nosotros y por tanto la solución puramente de cálculo no es accesible.

Por tanto, la única forma realmente potente y útil de encontrar una buena aproximación será incorporar el conocimiento a priori que tenemos del sistema. En la sección anterior hemos visto que una forma de incorporar dicho conocimiento es mediante la selección de la clase de funciones, pero además será muy relevante el hecho de cómo los datos son empleados en el proceso de aprendizaje. En este apartado de decisión tendremos que resolver primero la codificación de los datos, el algoritmo empleado y el uso de técnicas como la regularización que veremos después para incorporar nuestro conocimiento en el camino que nos lleve a la solución.

\subsection{Clases de aprendizaje}

El problema de aprendizaje puede ser subdividido a su vez en cuatro clases distintas y que se suelen abordar de forma independiente. Estoss tipos de problemas de aprendizaje son:

\begin{itemize}
	\item Clasificación: El problema de clasificación consiste en identificar y separar instancias de datos según su clase. Por ejemplo podemos dividir a la población mundial en dos clases: sanos y enfermos. Un problema de clasificación podría ser saber identificar estas clases para un conjunto de personas. Los problemas de clasificación más sencillos son aquellos en los que se usan dos únicas clases aunque se puede generalizar la definición del problema a k-clases.
	\item Regresión: El problema de regresión consiste en estimar una función $f: \mathbb{R}^n \rightarrow \mathbb{R}$ a partir de una serie de muestras previas con los valores de $f$. Un problema de regresión podría ser determinar la función que, dados los datos de altura y dimensiones corporales sea capaz de darnos el peso aproximado de la persona.
	\item Estimación de la función de densidad: en este caso no nos interesa la salida que proporciona el sistema, ya sea el valor de una clase o una función real como en el caso de la regresión. En este caso el objetivo del aprendizaje es conseguir la función de densidad $f(x,\omega)$, con $\omega \in \Omega$ los parámetros necesarios de la función de densidad, con la que se distribuyen los datos de entrada del sistema.
	\item Agrupamiento y cuantificación vectorial: El problema de cuantificación vectorial consiste en intentar explicar la distribución de los vectores de entrada mediante puntos clave llamados centroides. De esta forma se podría reducir la complejidad de los datos expresándolos en función de un sistema de generadores menor. El problema de agrupamiento tiene también relación por utilizar la idea de centroide, pero el objetivo es completamente distinto. El objetivo del problema de agrupamiento es intentar conseguir agrupar los datos en clusters, es decir, regiones del espacio en las que se concentran un conjunto de datos. De esta forma intentamos agrupar los datos que mantienen una relación entre sí. Un ejemplo de un problema de cuantificación vectorial podría ser un problema de reducción de dimensionalidad y un ejemplo de problema de agrupamiento podría ser identificar instancias de datos con características comunes.
\end{itemize}

\section{Principios y adaptación del aprendizaje}

Según Vapnik \cite{vapnik_v._nature_nodate} la predicción mediante el aprendizaje se puede dividir en dos fases:

\begin{enumerate}
	\item Aprendizaje o estimación a partir de una muestra.
	\item Predicción a partir de las estimaciones obtenidas.
\end{enumerate}

Estas dos fases se corresponden con los dos tipos de inferencia clásica que conocemos, esto es, inducción y deducción. Traído a este caso el proceso de inducción es aquel que a partir de los datos de aprendizaje o los datos de la muestra que tenemos con la salida que corresponde podemos estimar un modelo. Es decir, estamos sacando el conocimiento de los datos para generar el modelo. El proceso de deducción es aquel que, una vez obtenido el modelo estimado (la generalización) obtenemos una predicción de la salida sobre un conjunto de datos.

Por contra, Vapnik propone un paso que resuelve estas dos fases directamente y que él denomina transducción. Este paso consiste en, dados los datos de entrenamiento obtenemos directamente los valores de salida sin tener que hacer la generalización a un modelo. De esta forma, según Vapnik, podríamos reducir el error que cometemos en la predicción. Este razonamiento tiene sentido, pues estamos omitiendo el paso más complejo del proceso de inducción-deducción.

En resumen esta idea se puede resumir en la siguiente figura:

\begin{figure}[H]
	\centering
	\includegraphics[scale=0.5]{imagenes/induccion_deduccion_transduccion}
	\label{ind_ded_trans}
	\caption{Tipos de inferencia y transduccion \cite[p.~41]{cherkassky_learning_2007}}
\end{figure}

Podemos ver que el conocimiento a priori que tenemos del problema se manifiesta una vez se crea el modelo general, de forma que se emplearía en el paso de la inducción. Ya hemos hablado previamente del conocimiento a priori y cómo incorporarlo al modelo, pero por concretar un poco más podemos añadirlo básicamente de dos formas:

\begin{itemize}
	\item Escogiendo un conjunto de funciones para aproximar la salida del sistema
	\item Añadiendo restricciones o penalizaciones adicionales a dicho conjunto de funciones.
\end{itemize}

En resumen, para poder crear la generalización del modelo de forma única necesitamos:

\begin{enumerate}
	\item Un conjunto de funciones para aproximar la salida.
	\item Conocimiento a priori.
	\item Un principio inductivo, que no es más que una indicación de cómo emplear los datos para llegar a la generalización del modelo.
	\item Un método de aprendizaje, es decir, una implementación del principio inductivo.
\end{enumerate}

En secciones posteriores revisaremos algunos de los principios inductivos más usados pero es importante reseñar la diferencia entre pricipio inductivo y método de aprendizaje. Para un mismo principio inductivo podemos tener varios métodos de aprendizaje, pues podemos escoger diferentes formas de llevarlo a la práctica. Por ejemplo, uno de los principios inductivos más empleados es el ERM o Empirical Risk Minimization, es decir, minimización del error empírico. Podríamos pensar en diferentes formas de utilizar este principio, por ejemplo sólo avanzamos en la creación del modelo si a cada paso que demos minimizamos el error, o por ejemplo vamos avanzando varios modelos a la vez hasta obtener un número de modelos finales de entre los cuales escogeremos aquel que mejor minimice dicho error.

\subsection{Principios inductivos}

Una vez introducido el concepto como hemos hecho en la sección anterior vamos a hacer un breve repaso de los principios más usados y en qué consiste cada uno de ellos.

\subsubsection{Penalización o Regularización}

Imaginemos que tenemos una clase de funciones muy flexible, esto es con un gran número de parámetros libres $f(x,\omega)$ con $\omega \in \Omega$. Vamos a partir de la base del ERM, es decir, minimizar el error empírico. La penalización lo que va a hacer es añadir un factor a la función a minimizar:

$$R_{pen}(\omega) = R_{emp}(\omega) + \lambda \phi [f(x,\omega)]$$

Donde $R_{emp}(\omega)$ es el error empírico con los parámetros $\omega$ y $\phi [f(x,\omega)]$ es un funcional no negativo asociado a cada estimación $f(x,\omega)$. El parámetro $\lambda >0$ es un escalar que controla el peso de la penalización.

El funcional $\phi [f(x,\omega)]$ puede medir lo que creamos conveniente que debemos añadir, es decir, aquí podemos añadir a la minimización algún tipo de medida que nos diga cómo de bien funciona el ajuste de los datos y cómo de bien funciona la información a priori que hemos incluido en el modelo. Pensemos por ejemplo que $\lambda$ fuera un parámetro con un valor muy alto. En este caso la penalización por un mal ajuste de los datos no sería de gran importancia pues lo más conveniente sería minimizar el valor del funcional para no obtener una gran penalización. De esta forma podemos ajustar y dar un poco más de información al error empírico. Por ejemplo, en función del problema, es posible medir la complejidad de la solución mediante el funcional $\phi$ y de esta forma no sólo vamos a obtener una función que ajuste bien los datos, si no que también mantenga una cierta simplicidad para evitar por ejemplo el sobreajuste.

\subsubsection{Reglas de parada anticipada}

Pensemos en un método que vaya aprendiendo de los datos de forma iterativa intentando a cada iteración reducir el error cometido, por ejemplo el ERM. Los métodos o reglas de parada anticipada pueden verse como penalizaciones sobre el algoritmo conforme se va ejecutando. Las reglas de parada anticipada, como su nombre indica lo que preveen es la parada del algoritmo antes de obtener su objetivo teórico. Por ejemplo un algoritmo intenta que el error sea menor que $10^{-6}$ pero para reducirlo desde $10^{-4}$ hasta $10^{-5}$ está consumiendo millones de iteraciones. Si queremos que el tiempo de cómputo penalice lo que podemos hacer es fijar por ejemplo un número máximo de iteraciones que detenga el método aunque no se haya alcanzado esa barrera de error que se preveía.

\subsubsection{Minimización del riesgo estructural o SRM}

Para entender esta filosofía nos ponemos en la situación de que ya sabemos la clase de funciones con la que vamos a aproximar la salida del sistema, por ejemplo hemos escogido la clase de funciones polinómicas. Bajo esta clase de funciones podemos ordenar las funciones por complejidad, entendiendo por complejidad el número de parámetros de la función. Por ejemplo los polinomios de grado $m$ son de menor complejidad que los de grado $m+1$. De esta forma podemos pensar en una estructura de la clase de funciones de la forma:

$$S_0 \subset S_1 \subset S_2 \subset \cdots$$

Este parámetro de complejidad también puede ser un principio a minimizar para intentar conseguir una solución adecuada pero también simple. La generalización de la medida de complejidad para las clases de funciones es la conocida como dimensión VC o dimensión de Vapnik-Chervonenkis.

\subsubsection{Inferencia Bayesiana}

Este principio inductivo se utiliza en el problema de estimación de la función de densidad. El principio es utilizar la conocida fórmula de Bayes para hacer una estimación de la función de densidad empleando el conocimiento a priori que disponemos del problema. La forma en la que se emplea esta fórmula es de la siguiente:

$$P[modelo | datos] = \frac{P[datos | modelo] \cdot P[modelo]}{P[datos]}$$

donde $P[modelo]$ es la probabilidad a priori, $P[datos]$ es la probabilidad de los datos de entrenamiento y $P[datos | modelo]$ es la probabilidad de que los datos estén generados por el modelo.

\subsubsection{Descripción de mínima longitud}

La idea de este principio es la minimización de la longitud que se necesita emplear para describir un modelo y la correspondiente salida. Llamamos l a la longitud total:

$$l = L(modelo) + L(datos | modelo)$$

Esta medida puede ser vista como una medida de complejidad conjunta de todo el modelo.

\section{Regularización}

Por la importancia de este principio inductivo vamos a desarrollarlo un poco más, junto con el concepto de penalización, la selección de los modelos y la relación entre sesgo y varianza. Este último es un concepto muy relevante en cuanto al aprendizaje y que en nuestro caso, al no poseer la clasificación real tendremos que tenerlo en cuenta.

\subsection{Problema de la alta dimensionalidad}

Sabemos que cuando estamos ante un problema de aprendizaje nuestro objetivo es conseguir estimar una función con un número finito de instancias de una muestra ya con la salida. Al tener un número finito de elementos en la muestra ya sabemos que no podemos garantizar que la respuesta sea la óptima o correcta, pero además debemos pensar que a mayor regularidad del conjunto de funciones empleado debemos tener una densidad suficiente de puntos para compensar dicha regularidad. Este problema es conocido como la maldición de la dimensionalidad (curse of dimensionality). El problema es que cuanto mayor sea la dimensionalidad considerada más difícil es poder tener esa alta densidad de datos que se requieren para funciones muy regulares.

Este problema que conlleva la alta dimensionalidad proviene de la geometría de los espacio con alta dimensionalidad. A medida que incrementamos la dimensionalidad el espacio se ve cada vez con más aristas o picos. Podemos pensar en un cubo para el espacio tridimensional y a medida que aumentamos la dimensión incorporamos más aristas y vértices. Podemos resumir en 4 propiedades de los espacio con alta dimensionalidad que causan este problema:

\begin{enumerate}
	\item La densidad disminuye exponencialmente al aumentar el número de dimensiones. Supongamos que tenemos una muestra de $n$ puntos en $\mathbb{R}$. Para poder tener la misma densidad en un espacio $d$-dimensional $\mathbb{R}^d$ necesidamos $n^d$ puntos.
	\item Cuanto mayor dimensionalidad tenga el conjunto de datos mayor lado se necesita para que un hipercubo contenga el mismo porcentaje del conjunto que con una menor dimensionalidad. Imaginemos que tenemos un conjunto $d$-dimensional en el que tenemos la muestra dentro de un hipercubo unidad. Si quisiéramos abarcar un porcentaje $p\in [0,1]$ necesitaríamos un cubo de lado $e_d (p) = p^{\frac{1}{d}}$. Como se puede observar a mayor dimensionalidad y $p$ constante el lado es cada vez mayor. Esta idea es fácilemente entendible si observamos la siguiente figura:
	
	\begin{figure}[H]
		\centering
		\label{radio_alta_dimensionalidad}
		\includegraphics[scale=0.6]{imagenes/radio_alta_dimensionalidad}
		\caption{Para 2 dimensiones necesitamos menor lado que para 3 dimensiones. \cite[p.~64]{cherkassky_learning_2007}}
	\end{figure}
	\item Casi todo punto está más cerca de un borde que de otro punto. Pensemos en un conjunto de datos con $n$ puntos distribuidos de forma uniforme en una bola $d$-dimensional de radio unidad. Para este conjunto de datos, según \cite{hastie_t._elements_nodate}, la distancia media entre el centro de la distribución y los puntos más cercanos a dicho centro se mide bajo la fórmula:
	
	$$D(d,n) = (1-\frac{1}{2}^{1/n})^{1/d}$$
	
	Si en esta fórmula tomamos por ejemplo $n=200$ y $d=10$ el resultado es $D(10,200) \approx 0.57$. Esto significa que los puntos más cercanos al centro de la distribución están más cerca de los bordes que del centro.
	\item Casi todo punto es una anomalía sobre su propia proyección. Si pensamos de nuevo en la idea de los vértices y aristas en espacio de alta dimensionalidad y pensamos en que, según el punto anterior, cada vez que aumenta la dimensionalidad los puntos están más cerca de los bordes entonces no es extraño pensar que los puntos a medida que aumenta la dimensionalidad están más distantes del resto de puntos. Esto intuitivamente (ya que aún no hemos visto la definición formal de anomalía) nos guía a pensar que vistos los puntos en sus propios entornos éstos serán anomalías comparados con el resto.
	
	\begin{figure}[H]
		\centering
		\label{espacio_alta_dimension}
		\includegraphics[scale=0.6]{imagenes/espacio_alta_dimension}
		\caption{Forma conceptual de un espacio de alta dimensionalidad.\cite[p.~64]{cherkassky_learning_2007}}
	\end{figure}

	Conceptualmente podemos imaginarlo con esta forma de picos, con lo que si tenemos los datos apiñados en dichos picos o extremos el resto de datos que estén en picos diferentes distan tanto del que estamos considerando que no podemos afirmar que tengan ninguna relación entre sí.
\end{enumerate}

Estos puntos hemos de recordar que van referidos al conjunto de datos y no a las funciones que estamos considerando para representar la salida del sistema. Si estamos considerando la complejidad de las funciones la dimensionalidad no es una buena medida. Sabemos de la existencia de teoremas de aproximación de funciones como por ejemplo el Teorema de Superposición de Kolmogorov-Arnold.

\begin{teorema}[Teorema de Superposición de Kolmogorov-Arnold]
	Sea $f$ una función continua de varias variables $f:X_1 \times ... \times X_n \rightarrow \mathbb{R}$, entonces existen funciones $\Phi_q : \mathbb{R}\rightarrow \mathbb{R}$ y $\phi_{q,p} : X_p \rightarrow [0,1]$ tales que $f$ se puede expresar como:
	
	$$f(x) = f(x_1, ..., x_n) = \sum_{q=0}^{2n}\Phi_q ( \sum_{p=1}^{n}\phi_{q,p}(x_p))$$
\end{teorema}

Este Teorema argumenta perfectamente que la complejidad que le damos a los datos por tener una alta dimensionalidad no es transferible a las funciones pues podemos expresar funciones de varias variables como combinación de funciones de una sola variables. En otras palabras no podemos argumentar que la complejidad de funciones univariantes sea mayor o menor que la de funciones multivariantes.

\chapter{Concepto de anomalía}
\label{chapter:anomalia}
%
\chapter{Concepto de anomalía}
\label{chapter:anomalia}

\section{Contextualización}

Ya hemos discutido previamente una idea intuitiva del concepto de anomalía. Un dato decimos que es anómalo cuando se distancia del resto de los datos lo suficiente como para no tener características comunes con el resto.

Este hecho puede ser por distintos motivos. Puede que la anomalía venga del hecho de que se está produciendo un evento en nuestros experimentos que no sea nada frecuente. Por ejemplo podemos estar midiendo datos meteorológicos y que en un momento dado se den una serie de fenómenos que no sea frecuente ver juntos, o incluso que no se hayan visto nunca ocurrir simultáneamente. Otra forma de tener una anomalía en nuestro conjunto de datos pudiera ser errores de medición. Por ejemplo si seguimos con este símil de los datos meteorológicos imaginemos que nuestra estación dispone de un termómetro. Este sensor se ha roto y empieza a marcar datos superiores a 100 $C^\circ$, claramente son datos muy desviados de las temperaturas normales con lo que no tendrían relación con el resto y presentaría una desviación muy importante con respecto al resto de los datos.

\section{Criterios}

Esta idea intuitiva que estamos dando de anomalía no refleja todos los posibles escenarios. Los ejemplos que estamos dando suponen una desviación muy grande de los datos normales, tanto que no se pueden comparar con el resto porque difieren mucho numéricamente. Vamos a plantear un escenario para dar una mejor forma al concepto de anomalía. Pensemos en una serie de datos muy agrupados en dos clústers por ejemplo:

\begin{figure}[H]
	\centering
	\includegraphics[scale=0.5]{imagenes/clusters}
	\label{clusters}
	\caption{Clusters alejados}
\end{figure}

Como podemos comprobar que tenemos dos clústers no sólo alejados entre sí, si no con los elementos muy concentrados para poner un caso extremo. Ahora no vamos a proponer un valor que se aleje de los dos clústers, si no uno que esté a medio camino entre los dos:

\begin{figure}[H]
	\centering
	\includegraphics[scale=0.5]{imagenes/outlier_cluster}
	\label{outlier_clusters}
	\caption{Clusters alejados con una anomalía en rojo}
\end{figure}

Si los datos del clúster de abajo a la izquierda fueran datos de temperatura con valores entorno a 0 y los de arriba a la derecha fueran de datos de temperatura entorno a 100 grados nuestro datos anómalo tendría una temperatura de unos 50 grados. Esta temperatura no se aleja radicalmente de los valores normales, es decir, no son -1000 grados ni 1000 grados. Aún así estamos describiendo una situación anómala.

No podemos dar una definición formal o que podamos decir que abarca todos los casos para definir lo que es una anomalía, aún así vamos a intentar dar dos puntos de vista: uno basado en distancias y otro en probabilidades.

El criterio más usado en la definición o detección de anomalías es el llamado ``Tukey's Fences''. Para introducirlo vamos a ver su definición en una única dimensión para luego extender el concepto. Pensemos en un conjunto de datos 1-D. Sobre sus valores podemos calcular los cuartiles $Q_1 , Q_2 $ y $Q_3$. Un valor anómalo es aquel que no cae dentro del intervalo $[Q_1 - k(Q_3 - Q_1), Q_3 + k(Q_3 - Q_1)]$ donde $k$ es una constante. El valor propuesto para $k$ por Tukey fue de $k=1.5$ aunque algunos autores más restrictivos proponen $k=3$.

Este criterio puede ser extendido al caso de mayor dimensionalidad si realizamos este mismo test sobre todos los valores de todas las características y comprobar si alguno o todos se salen del rango en función de cómo de restrictivo queremos que sea el criterio.

Esta extensión es muy vaga, por lo que se propone un criterio un poco más fijado. Imaginemos los datos agrupados por clústers, entonces podemos fijar un centroide de dicho cluster. Sobre cada cluster podemos medir cuál es la mayor distancia intercluster de los datos al centroide. Podemos extender el criterio de Tukey diciendo que un dato anómalo es aquel que se distancia más de $1.5$ veces de la mayor distancia intercluster al centroide.

Esta generalización ya si abarca el ejemplo que hemos propuesto. Al estar muy apiñados los datos entorno al centroide la mayor distancia intercluster es muy pequeña, de hecho en el ejemplo construido es menor que 5. Por tanto el dato $(50,50)$ está alejado más de $1.5 \cdot 5 = 7.5$ unidades del centroide y por tanto lo podemos considerar una anomalía.
%
\chapter{Introducción de Estadística Multivariante}
\label{chapter:estadistica_multivariante}

Vamos a dar otra definición de anomalía que no coincide con la que hemos visto basada en distancias, pero antes de dar esa definición debemos hacer un breve repaso de estadística multivariante y probabilidad para poder comprender y enmarcar dicha definición.

\section{Introducción}

En primer lugar vamos a describir conceptos básicos sobre los que poder construir los conceptos que necesitamos para la definición de anomalía basada en probabilidades.

En primer lugar vamos a definir el concepto de variable aleatoria.

\begin{definicion}
	Una variable aleatoria es una función $X:\Omega \rightarrow E$ que parte de un espacio de probabilidad $(\Omega , \mathcal{F}, \mathcal{P})$ y llega a un espacio medible $(E, \mathcal{B})$, donde $X$ además es una función medible.
\end{definicion}

Normalmente ya sabemos que $E\subseteq \mathbb{R}$ y además cabe recordar que $\mathcal{F}$ es una $\sigma$-álgebra. Además cabe recordar la definición de función medible:

\begin{definicion}
	Decimos que una función $X: (\Omega , \mathcal{F}, \mathcal{P}) \rightarrow (E, \mathcal{B})$ es medible si $X^{-1}(B)\subset \mathcal{F}$, $\forall B \in \mathcal{B}$.
\end{definicion}

Esta definición puede extenderse al caso vectorial, introduciendo con esto la noción de vector aleatorio:

\begin{definicion}
	Un vector aleatorio $\underline{X} = (X_1 , ... , X_p)$ es una aplicación medible $\underline{X}: (\Omega , \mathcal{F}, \mathcal{P})\rightarrow (E, \mathcal{B}^p)$ donde $E\subseteq \mathbb{R}^p$.
\end{definicion}

Se puede demostrar además la caracterización:

\begin{proposicion}
	Un vector $\underline{X} = (X_1, ..., X_p)$ es un vector aleatorio si y sólo si $X_i : (\Omega , \mathcal{F}, \mathcal{P}) \rightarrow (\mathbb{R}, \mathcal{B})$ es una función medible.
\end{proposicion}

Con este vector aleatorio podemos estudiar o definir la distribución de probabilidad del mismo sobre $( \mathbb{R}^p , \mathcal{B}^p )$ $P_{\underline{X}}$ como:

$$P_{\underline{X}} [B]:= P[\underline{X}^{-1}(B)] \ \forall B\in \mathcal{B}$$

con lo que el espacio $(\mathbb{R}^p , \mathcal{B}^p , P_{\underline{X}})$ es un espacio de probabilidad o probabilístico.

Sobre los conocimientos de la definición de la función de distribución univariante podemos hacer una definición análoga para el caso multivariante.

\begin{definicion}
	Se define la función de distribución asociada a la probabilidad inducida como:
	
	$$F_{\underline{X}} (\underline{x}) = P_{\underline{X}} [X_1 \leq x_1 , ... , X_p \leq x_p] \ , \ \forall \underline{x} = (x_1 , ... , x_p) \in \mathbb{R}^p$$
\end{definicion}

De igual forma podemos caracterizar la función de densidad como aquella $f_{\underline{X}}$ que, de existir, cumple que:

$$F_{\underline{X}} (\underline{x}) = \int_{- \infty}^{x_1} \int_{-\infty}^{x_2} ... \int_{-\infty}^{x_p} f_{\underline{X}}(u_1 , ... , u_p) du_1 ... du_p$$

Otra forma de determinar de forma única la distribución de un vector aleatorio es mediante la función característica, lo que nos va a dar además una caracterización de la independencia que introduciremos en siguiente lugar.

\begin{definicion}
	Dado un vector aleatorio $X = (X_1 , ... , X_p)$ se define la función característica como $\Phi_{\underline{X}} (\underline{t}) = E[e^{i\underline{t}X}]$ con $\underline{t} = (t_1 , ... , t_p)\in \mathbb{R}^p$ donde la función $E[\cdot]$ denota la esperanza, por lo que:
	$$\Phi_{\underline{X}} (\underline{t}) = \int_{\mathbb{R}^p} e^{i\underline{t} \underline{X}} P_{\underline{X}}(d\underline{x})$$
\end{definicion}

Con esto ya podemos introducir el concepto de independencia en varias variables. 
%
\chapter{Concepto probabilístico de anomalía}
\label{chapter:anomalia_probabilidad}

Esta introducción de probabilidad nos va a servir tanto para la definición de anomalía alternativa a la basada en distancias que vamos a ver como para la explicación y análisis de los modelos.

En primer lugar cabe decir que esta definición, al igual que el criterio ya explicado no engloba todas las anomalías y por tanto es algo difícil de medir. Esta definición hace referencia, según mi criterio, a un enfoque que se debe poner junto a la definición basada en distancias y no en contraposición. El objetivo de esta definición es obtener anomalías que no son triviales y se esconden entre los datos.

La base del razonamiento de este tipo de anomalías surge del hecho de que un objeto puede ser anómalo en un subespacio concreto de los datos, pero no en el espacio total. Vamos a introducir un ejemplo para visualizar un tipo de anomalía que encaje con esta definición.

Veamos la siguiente figura:

\begin{figure}[H]
	\label{ejemplo_anomalia_probabilidad}
	\includegraphics[scale=0.6]{imagenes/ejemplo_anomalia_probabilidad}
	\caption{Ejemplo de anomalía \cite{fabian_keller_hics:_2012}}
\end{figure}

Como se puede observar tenemos dos espacios: el izquierdo no presenta datos correlados y el derecho sí presenta correlación. Podemos ver que en ambos casos se comparte una anomalía etiquetada como $O_1$. Esta anomalía en el caso del espacio no correlado es perfectamente detectable de forma trivial observando las proyecciones de los datos en una dimensión. En cambio en el segundo caso ninguna de las dos anomalías etiquetadas $O_1 , O_2$ son detectables de esta forma trivial, pues si hacemos las proyecciones uno dimensionales ninguno de los dos datos es discordante en dichas proyecciones. Estas anomalías son las que decimos que son no triviales. En cambio si observamos los datos en una proyección de orden superior como la que estamos viendo de dimensión 2 podemos observar claramente que se salen de la correlación de datos que muestra el resto. Es aquí donde podemos ver que en el conjunto de la derecha ninguno de los puntos es una anomalía en las proyecciones de dimensión uno pero sí lo son en la proyección de dimensión 2.

Vamos por tanto a definir más formalmente este concepto especial de anomalía. Necesitamos introducir en primer lugar un poco de notación.

Partimos de un conjunto de datos $X = \{ x_1 , ... , x_n \}$ de $n$ objetos cada uno tomando $d$ valores, es decir, $x_i = (x_{s_1} , ... , x_{s_d}) \in \mathbb{R}^d$. Notamos un subespacio del conjunto de valores como:

$$S = \{ s_i | s_i \in \{ s_1 , ... , s_d \} \ con \ i\in \Delta \}$$

Dado un subespacio $S = \{ s_1 , ... , s_p \}$ notamos la proyección de los objetos del conjunto de datos como $X_{S} = \{ x_{s_1} , ... , x_{s_p} \}$.

Esta proyección está distribuida según una distribución conjunta desconocida de $S$:

$$p_{s_1 , ... , s_p} (x_{s_1} , ... , x_{s_p})$$

Notamos la distribución marginal asociada al atributo $s_i$ como:

$$p_{s_i}(x_{s_i})$$

\begin{definicion}
	Decimos que un subespacio $S$ es un espacio incorrelado si y sólo si:
	
	$$p_{s_1 , ... , s_p}(x_{s_1} , ... , x_{s_p}) = \prod_{i=1}^{p}p_{s_i}(x_{s_i})$$
\end{definicion}

Por tanto si estamos bajo la suposición de un espacio incorrelado podemos decir que la densidad esperada es:

$$p_{esp}(x_{s_1} , ... , x_{s_p}) \equiv \prod_{i=1}^{p}p_{s_i}(x_{s_i})$$

Recordemos que nuestras anomalías no triviales no están en este tipo de subespacios, si no en los correlados. Por tanto vamos a definirlo de la siguiente forma:

\begin{definicion}
	Decimos que un objeto $x_{S}$ es una anomalía no trivial respecto al subespacio $S$ si:
	
	$$p_{s_1 , ... , s_p}(x_{s_1} , ... , x_{s_p}) \ll p_{esp}(x_{s_1} , ... , x_{s_p})$$
	
	Es decir, si la probabilidad esperada es significativamente mayor que la probabilidad conjunta.
\end{definicion}

Por cómo hemos definido los espacios correlados e incorrelados es claro que no podemos tener anomalías en espacios no correlados como es evidente pues la densidad conjunta y esperada serían iguales.

Este concepto como podemos observar no comparte ninguna relación con nuestra definición de anomalías basadas en distancias por lo que es de esperar que si comparamos ambos tipos de anomalías en un conjunto de datos no obtengamos los mismos objetos.
%
\chapter{Modelos implementados}
\label{chapter:modelos}

En este capítulo vamos a repasar qué modelos he implementado y cómo funcionan cada uno de ellos. Primero se hará una revisión teórica de los modelos y posteriormente un análisis breve del código explicando las particularidades de las implementaciones.

\section{Algoritmos de ensamblaje}

Los algoritmos que he implementado pertenecen a una familia concreta de algoritmos de detección de anomalías denominados como algoritmos de ensamblaje o ``Ensemble Algorithms'' en inglés. Estos algoritmos son lo equivalente a los meta-algoritmos pero destinados a la detección de anomalías. Para dar una mejor definición de qué son los algoritmos de ensamblaje vamos a introducir una clasificación de los mismos para dar las categorías que entran dentro de esta definición.

\begin{itemize}
	\item Algoritmos de ensamblaje secuenciales: En este tipo de algoritmos tenemos un algoritmos base o un conjunto de algoritmos base que se aplican de forma secuencial, de forma que las primeras ejecuciones se ven usadas o modificadas por ejecuciones futuras de algoritmos. Finalmente el resultado puede ser una combinación ponderada de las valoraciones de los algoritmos o el resultado del último de ellos.
	
	
	\begin{algorithm}[H]{\textbf{Ensamblaje secuencial:}}
		\SetAlgoLined
		
		\textbf{Entrada: } Conjunto de datos $\mathcal{D}$, Algoritmos base $\mathcal{A}_1 , ... , \mathcal{A}_r$
		
		j=1
		
		\Repeat{fin}{
			Tomamos el algoritmo $\mathcal{A}_j$ según los resultados anteriores
			
			Tomamos el conjunto de datos modificado $f_j (\mathcal{D})$ de anteriores ejecuciones
			
			Ejecutamos el algoritmo $\mathcal{A}_j$ sobre $f_j (\mathcal{D})$
			
			j=j+1
			
		}
	
		\KwResult{Combinación de los resultados}
	\end{algorithm}
	\item Algoritmos de ensamblaje independientes: En este caso se emplean o bien diferentes instancias del mismo algoritmo o bien diferentes porciones de los datos que se emplearán de forma distinta. Se puede variar la instanciación por ejemplo dependiendo del subespacio sobre el que queramos ejecutarlo o dependiendo de las características de una porción concreta de los datos.
	
	\begin{algorithm}[H]{\textbf{Ensamblaje independiente:}}
		\SetAlgoLined
		
		\textbf{Entrada: } Conjunto de datos $\mathcal{D}$, Algoritmos base $\mathcal{A}_1 , ... , \mathcal{A}_r$
		
		j=1
		
		\Repeat{fin}{
			Tomamos el algoritmo $\mathcal{A}_j$
			
			Creamos el conjunto de datos modificado $f_j (\mathcal{D})$
			
			Ejecutamos el algoritmo $\mathcal{A}_j$ sobre $f_j (\mathcal{D})$
			
			j=j+1
			
		}
		
		\KwResult{Combinación de los resultados}
	\end{algorithm}
\end{itemize}

\section{Mahalanobis Kernel}

Este algoritmo está englobado dentro de la categoría de algoritmos basados en dependencia. Esta clase de algoritmos intenta estudiar las dependencias que existen entre atributos para así poder detectar las instancias u objetos que no tienen estas dependencias y marcarlos como anomalías.

Si intentamos visualizar esta dependencia entre atributos de forma gráfica lo que observaríamos es que los datos están alineados o posicionados en hiperplanos lineales o no lineales de la siguiente forma:

\begin{figure}[H]
	\centering
	\label{hiperplano}
	\includegraphics[scale=0.8]{imagenes/hiperplano}
	\caption{Hiperplano}
\end{figure}

Esta figura es un ejemplo clásico de estudio de algoritmos como por ejemplo PCA (algoritmo que quedaría dentro de esta categoría).

\begin{figure}[H]
	\centering
	\label{hiperboloide}
	\includegraphics[scale=2.5]{imagenes/hiperboloide}
	\caption{Hiperboloide \href{https://commons.wikimedia.org/wiki/File:Circular_Hyperboloid_Of_One_Sheet_Quadric.png}{Wikimedia}}
\end{figure}

En este caso tenemos el ejemplo de un hiperboloide que no tiene una dependencia lineal, si no que presenta una dependencia de tipo cuadrático.

El método de Mahalanobis Kernel puede ser visto como una modificación de PCA. PCA básicamente dispone de dos pasos:

\begin{enumerate}
	\item Determinar un sistema ortogonal de direcciones principales y proyectar los datos sobre este sistema.
	\item Calcular la distancia entre el punto original y la proyección como su puntuación de anomalía.
\end{enumerate}

El método Mahalanobis Kernel intenta tener este mismo comportamiento en dos pasos y que ahora veremos. El algoritmo PCA es muy útil cuando los datos tienen atributos relacionados en un hiperplano, mientras que Mahalanobis Kernel funciona mejor cuando los datos están relacionados en formas más complejas como el hiperboloide que hemos enseñado. La elección de este algoritmo en vez de PCA recae en el hecho de que PCA es un algoritmo clásico y el escenario en el que mejor funciona (hiperplano) es más restrictivo que el que nos ofrece Mahalanobis Kernel con un abanico de figuras más amplio.

Vamos a describir el funcionamiento del algoritmo, pero primero vamos a introducir notación. Vamos a llamar $D$ a la matriz de datos que está centrada en la media y que tiene dimensiones $n\times d$, es decir, tenemos $n$ instancias u objetos de dimensionalidad $d$.

\begin{algorithm}{\textbf{Mahalanobis Kernel}}
	\caption{Mahalanobis Kernel}
	\label{mahalanobis_kernel}
	\KwIn{$D$}
	
	$S = DD^T$.
	
	$S = Q\Delta^2 Q^T$.
	
	Almacenamos los vectores propios columna no negativos de $Q\Delta$ en una matriz $D'$
	
	Normalizamos $D'$ para que tenga media $0$ y varianza $1$.
	
	$vector\_media = media(D')$
	
	$puntuaciones = []$
	
	\ForEach{fila en D'}{
		$score = distancia(vector\_media , fila)$
		
		$puntuaciones = [puntuaciones, score]$
	}
	\KwOut{puntuaciones}
\end{algorithm}

El algoritmo comienza con la matriz de datos $D$. Se obtiene la matriz simétrica $S$ y se hace la descomposición en valores singulares. 

Con este modelo tenemos las dos fases que teníamos en PCA. Primero obtenemos una matriz $D'$ de los datos proyectados y transformados para posteriormente reportar la puntuación de anomalía como una distancia.

Veamos ahora la implementación en Python.

\begin{lstlisting}[language=Python]
def runMethod(self):
'''
@brief Function that executes the Kernel Mahalanobis method. The results are
stored on the variable self.scores
@param self
'''
''' Compute the S matrix of the algorithm'''
S = np.dot(self.dataset, self.dataset.T)
''' Now we diagonalize it'''
Q,delta_sq,Qt = np.linalg.svd(S)
del S
del Qt
''' Obtain delta as matrix'''
delta = np.matrix(np.diag(np.sqrt(delta_sq)))
del delta_sq
Q = np.matrix(Q)
''' Compute de D' matrix and normalize it'''
Dprime = np.dot(Q,delta)
del Q
del delta
Dp_std = scale(Dprime, axis=1)
del Dprime
''' We compute its mean on the rows to compute the deviation as the score'''
mean = Dp_std.mean(axis=0)
self.outlier_score=[]
''' The score is the euclidean distance to the mean'''
for i in range(len(Dp_std)):
self.outlier_score.append(np.linalg.norm(mean-Dp_std[i])**2)
self.outlier_score = np.array(self.outlier_score)
self.calculations_done=True
\end{lstlisting}

La implementación del algoritmo se ha realizado en Python como el resto del proyecto y posteriormente se explicará en detalle cómo se ha organizado.

El algoritmo basa su implementación en la librería NumPy.
%
%\input{capitulos/07_Pruebas}
%
%\input{capitulos/08_Conclusiones}
%
%%\chapter{Conclusiones y Trabajos Futuros}
%
%
\nocite{*}
\bibliography{referencias}\addcontentsline{toc}{chapter}{Bibliografía}
\bibliographystyle{unsrt}
%
%\appendix
%\input{apendices/manual_usuario/manual_usuario}
%%\input{apendices/paper/paper}
%\input{glosario/entradas_glosario}
% \addcontentsline{toc}{chapter}{Glosario}
% \printglossary
\chapter*{}
\thispagestyle{empty}

\end{document}
